\subsection*{Цель работы}

Изучить методы и последовательность синтеза

\subsection*{1. Таблица истинности}

$\begin{matrix}
    X_1 & 0 & 0 & 0 & 0 & 1 & 1 & 1 & 1\\
    X_2 & 0 & 0 & 1 & 1 & 0 & 0 & 1 & 1\\
    X_3 & 0 & 1 & 0 & 1 & 0 & 1 & 0 & 1\\
    y_1 & 0 & 1 & 1 & 0 & 1 & 0 & 0 & 1\\
    y_2 & 0 & 0 & 0 & 1 & 0 & 1 & 1 & 1
\end{matrix}$

\subsection*{2. Переключательная функция в СДНФ}

$y_1 = \overline{X_1} \overline{X_2} X_3 \lor \overline{X_1} X_2 \overline{X_3} \lor X_1 \overline{X_2} \overline{X_3} \lor X_1 X_2 X_3$

$y_2 = \overline{X_1} X_2 X_3 \lor X_1 \overline{X_2} X_3 \lor X_1 X_2 \overline{X_3} \lor X_1 X_2 X_3$

\subsection*{3. Минимизация функций}

Как видно из рисунка \ref{fig:karnaugh}, функция $y_1$ минимизации не подлежит.

\kvunitlength=13mm
\begin{figure}[ht]
    \subfloat[]{
        \karnaughmap{3}{$y_1:$}{{$X_1$}{$X_2$}{$X_3$}}{01101001}{}
    }
    \qquad
    \subfloat[]{
        \karnaughmap{3}{$y_2:$}{{$X_1$}{$X_2$}{$X_3$}}{00010111}{
            \thicklines
            \put(2,0.5){\color{green}\oval(1.9,0.9)}
            \put(3,0.5){\color{red}\oval(1.9,0.8)}
            \put(2.5,1){\color{blue}\oval(0.9,1.9)}
        }
    }
    \caption{Карты Карно (а) функции $y_1$; (б) функции $y_2$}
    \label{fig:karnaugh}
\end{figure}

$y_2 = X_2 X_3 \lor X_1 X_2 \lor X_1 X_3$

\clearpage
\subsection*{4. Перевод в базис Пирса}

$y_1 = \overline{X_1} \overline{X_2} X_3 \lor \overline{X_1} X_2 \overline{X_3} \lor X_1 \overline{X_2} \overline{X_3} \lor X_1 X_2 X_3$

$y_1 = \overline{(X_1 \downarrow X_2 \downarrow \overline{X_3}) \downarrow (X_1 \downarrow \overline{X_2} \downarrow X_3) \downarrow (\overline{X_1} \downarrow X_2 \downarrow X_3) \downarrow (\overline{X_1} \downarrow \overline{X_2} \downarrow \overline{X_3})}$

$y_2 = X_2 X_3 \lor X_1 X_2 \lor X_1 X_3$

$X_2 X_3 = \overline{X_2} \downarrow \overline{X_3}$

$X_1 X_2 = \overline{X_1} \downarrow \overline{X_2}$

$X_1 X_3 = \overline{X_1} \downarrow \overline{X_3}$

$y_2 = \overline{ (\overline{X_2} \downarrow \overline{X_3}) \downarrow (\overline{X_1} \downarrow \overline{X_2}) \downarrow (\overline{X_1} \downarrow \overline{X_3}) }$

\subsection*{5. Составление схемы}

\begin{figure}[ht]
    \begin{tikzpicture}[circuit logic IEC,thick]
        \node [not gate] (x1) at (0,4) {};
        \node [not gate] (x2) at (0,2) {};
        \node [not gate] (x3) at (0,0) {};
        \node [nand gate, inputs=nnn] (x1x2nx3) at (4,5) {};
        \node [nand gate, inputs=nnn] (x1x3nx2) at (4,3) {};
        \node [nand gate, inputs=nnn] (x2x3nx1) at (4,1) {};
        \node [nand gate, inputs=nnn] (nx1nx2nx3) at (4,-1) {};
        \node [nand gate, inputs=nnnn] (all) at (6,2){};
        \node [not gate] (end) at (8,2){};

        \node (c) at ([xshift=-20mm]x1.west) {$X_1$};
        \draw (x1.input) -- (c);
        \draw (x2.input) -- (x2.input-|c.east) node[left] (x2-label) {$X_2$};
        \draw (x3.input) -- (x3.input-|c.east) node[left] (x3-label) {$X_3$};

        \draw[red] ([xshift=-12mm]x1.west) |- (1,-3) -- (1,6) node[above] (x1label2) {$X_1$};
        \draw[green] ([xshift=-9mm]x2.west) |- (1.8,-2.5) -- (1.8,6) node[above] (x2label2) {$X_2$};
        \draw[blue] ([xshift=-6mm]x3.west) |- (2.6,-2) -- (2.6,6) node[above] (x2label2) {$X_2$};

        \draw (x1.output) --++(0:3mm) |- (x2x3nx1.input 1);
        \draw (x1.output) --++(0:3mm) |- (nx1nx2nx3.input 1);

        \draw (x2.output) --++ (0:10mm) |- (x1x3nx2.input 1);
        \draw (x2.output) --++ (0:10mm) |- (nx1nx2nx3.input 2);

        \draw (x3.output) --++ (0:17mm) |- (x1x2nx3.input 1);
        \draw (x3.output) --++ (0:17mm) |- (nx1nx2nx3.input 3);

        \draw[red] (x1x2nx3.input 2) --++ (-27mm,0);
        \draw[green] (x1x2nx3.input 3) --++ (-19mm,0);

        \draw[red] (x1x3nx2.input 2) --++ (-27mm,0);
        \draw[blue] (x1x3nx2.input 3) --++ (-11mm,0);

        \draw[green] (x2x3nx1.input 2) --++ (-19mm,0);
        \draw[blue] (x2x3nx1.input 3) --++ (-11mm,0);

        \draw (x1x2nx3.output) --++ (8mm,0) |- (all.input 1);
        \draw (x1x3nx2.output) --++ (5mm,0) |- (all.input 2);
        \draw (x2x3nx1.output) --++ (5mm,0) |- (all.input 3);
        \draw (nx1nx2nx3.output) --++ (8mm,0) |- (all.input 4);

        \draw(all.output) -- (end.input);
        \draw(end.output) --++(0:5mm) node[right] {$y_1$};
    \end{tikzpicture}
    \caption{Логическая схема $y_1$}
    \label{fig:y1}
\end{figure}

\begin{figure}[ht]
    \begin{tikzpicture}[circuit logic IEC,thick]
        \node [not gate] (x1) at (0,4) {};
        \node [not gate] (x2) at (0,2) {};
        \node [not gate] (x3) at (0,0) {};
        \node [xnor gate] (x1x2) at (2,4) {};
        \node [xnor gate] (x1x3) at (2,2) {};
        \node [xnor gate] (x2x3) at (2,0) {};
        \node [xnor gate, inputs=nnn] (all) at (4,2){};
        \node [not gate] (end) at (6,2){};

        \node (c) at ([xshift=-10mm]x1.west) {$X_1$};
        \draw (x1.input) -- (c);
        \draw (x2.input) -- (x2.input-|c.east) node[left] (x2-label) {$X_2$};
        \draw (x3.input) -- (x3.input-|c.east) node[left] (x3-label) {$X_3$};

        \draw[red] (x1.output) --++(0:7mm) |- (x1x2.input 1);
        \draw[red] (x1.output) --++(0:7mm) |- (x1x3.input 1);

        \draw[blue] (x2.output) --++(0:3mm) |- (x1x2.input 2);
        \draw[blue] (x2.output) --++(0:3mm) |- (x2x3.input 1);

        \draw[green] (x3.output) --++(0:7mm) |- (x1x3.input 2);
        \draw[green] (x3.output) --++(0:7mm) |- (x2x3.input 2);

        \draw(x1x2.output) -- ++(0:5mm) |- (all.input 1);
        \draw(x1x3.output) -- ++(0:5mm) |- (all.input 2);
        \draw(x2x3.output) -- ++(0:5mm) |- (all.input 3);

        \draw(all.output) -- (end.input);
        \draw(end.output) --++(0:5mm) node[right] {$y_1$};
    \end{tikzpicture}
    \caption{Логическая схема $y_2$}
    \label{fig:y1}
\end{figure}

\clearpage
