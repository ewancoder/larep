\section*{Введение}

В реферате рассматривается обработка детали, эскиз которой представлен на рисунке~\ref{fig:sketch}. Набор инструментов для обработки представлен на рисунке~\ref{fig:tools}.

Операционные эскизы деталей с обозначением соответствующих точек обработки представлены в приложениях А и Б соответственно. Там же представлен список точек и их координат.

\begin{figure}[ht]
    \includegraphics[width=.8\linewidth]{Figures/sketch.png}
    \caption{Эскиз изготовляемой детали}
    \label{fig:sketch}
\end{figure}

\begin{figure}[ht]
    \includegraphics[width=.5\linewidth]{Figures/tools.png}
    \caption{Набор инструментов}
    \label{fig:tools}
\end{figure}

\section{Программа обработки}

\subsection{Программа обработки детали с правой стороны}

Операционный эскиз обработки детали с правой стороны изображен в приложении А.

\begin{minted}[gobble=4, fontsize=\footnotesize, breaklines]{python}
    O1000 #Заголовок программы
    G21 #Работа в метрической системе
    G00 G18 G40 G49 G80 G90 #Строка безопасности
    G50 S4000 #Ограничение максимальной скорости вращения шпинделя 4000 об/мин
    T1 M6 #Выбор инструмента №1 (Проходной резец)
    G54 G92 Z100 S1200 M3 #Установка нуля в точку 0, вращение шпинделя по часовой стрелке со скоростью 1200 об/мин
    G43 H01 Z15 X70 #Коррекция на длину №1 с перемещением в точку 1
    G00 Z2 X62 #Перемещение в точку 2 (холостой ход)
    M08 #Включить СОЖ
    G84 Z-60 X20 D3=2000 D0=1000 D2=1000 #Цикл продольного точения до точки 3
    M09 #Выключить СОЖ
    G00 Z15 X70 #Возврат в точку 1 (холостой ход)

    T2 M6 #Выбор инструмента №2 (Проточной резец)
    G43 H02 Z15 X70 #Коррекция на длину №2 с перемещением в точку 1
    G00 Z0 X32 #Перемещение в точку 4 (холостой ход)
    M08 #Включить СОЖ
    G86 Z-10 X20 D3=2500 D4=10 D5=2000 F100 #Цикл нарезания канавок до точки 5
    G00 Z-30 #Перемещение в точку 6 (холостой ход)
    G86 Z-60 X20 D3=2500 D4=10 D5=2000 F100 #Цикл нарезания канавок до точки 7
    M09 #Выключить СОЖ
    G00 Z15 X70 #Возврат в точку 1 (холостой ход)

    T3 M6 #Выбор инструмента №3 (Фреза)
    G43 H03 Z15 X70 #Коррекция на длину №3 с перемещением в точку 1
    G00 Z-58 X40 #Перемещение в точку 8 (холостой ход)
    M08 #Включить СОЖ
    G01 Z-70 F100 #Линейная интерполяция в точку 9 с подачей 100 мм/мин
    M09 #Выключить СОЖ
    G00 Z-58 #Возврат в точку 8 (холостой ход)
    G00 Z15 X70 #Возврат в точку 1 (холостой ход)

    T4 M6 #Выбор инструмента №4 (Сверло)
    G43 H04 Z15 X0 #Коррекция на длину №4 с перемещением в точку 9
    G00 Z2 #Перемещение в точку 10 (холостой ход)
    M08 #Включить СОЖ
    G87 Z-30 D3=10000 D4=10 D5=70 D6=1000 F100 #Цикл сверления со стружколоманием до точки 11
    M09 #Выключить СОЖ
    G00 Z15 #Возврат в точку 9 (холостой ход)

    T5 M6 #Выбор инструмента №5 (Метчик)
    G43 H05 Z15 X0 #Коррекция на длину №5 с перемещением в точку 9
    G00 Z2 X5 #Перемещение в точку 12 (холостой ход)
    M08 #Включить СОЖ
    G01 Z-20 F40 #Линейная интерполяция в точку 13 с подачей 40 мм/мин
    M09 #Выключить СОЖ
    G00 Z15 X0 #Возврат в точку 9 (холостой ход)

    G56 G53 M30 T0000 #Конец программы
\end{minted}

\subsection{Программа обработки детали с левой стороны}

Операционный эскиз обработки детали с левой стороны изображен в приложении Б.

\begin{minted}[gobble=4, fontsize=\footnotesize, breaklines]{python}
    O1000 #Заголовок программы
    G21 #Работа в метрической системе
    G00 G18 G40 G49 G80 G90 #Строка безопасности
    G50 S4000 #Ограничение максимальной скорости вращения шпинделя 4000 об/мин
    T1 M6 #Выбор инструмента №1 (Проходной резец)
    G54 G92 Z100 S1200 M3 #Установка нуля в точку 0, вращение шпинделя по часовой стрелке со скоростью 1200 об/мин
    G43 H01 Z15 X70 #Коррекция на длину №1 с перемещением в точку 1
    G00 Z2 X62 #Перемещение в точку 2 (холостой ход)
    M08 #Включить СОЖ
    G84 Z-20 X20 D3=2000 D0=1000 D2=1000 #Цикл продольного точения до точки 3
    M09 #Выключить СОЖ
    G00 Z15 X70 #Возврат в точку 1 (холостой ход)

    T4 M6 #Выбор инструмента №4 (Сверло)
    G43 H04 Z15 X0 #Коррекция на длину №4 с перемещением в точку 4
    G00 Z2 #Перемещение в точку 5 (холостой ход)
    M08 #Включить СОЖ
    G87 Z-40 D3=10000 D4=10 F100 D6=1000 F100 #Цикл сверления со стружколоманием до точки 6
    M09 #Выключить СОЖ
    G00 Z15 #Возврат в точку 4 (холостой ход)

    T6 M6 #Выбор инструмента №6 (Расточной резец)
    G43 H06 Z15 X0 #Коррекция на длину №5 с перемещением в точку 4
    G00 Z2 X-5 #Перемещение в точку 7 (холостой ход)
    M08 #Включить СОЖ
    G84 Z-40 X-10 D3=2000 D0=1000 D2=1000 #Цикл продольного точения (расточка) до точки 8
    M09 #Выключить СОЖ
    G00 Z15 X0 #Возврат в точку 4 (холостой ход)

    G56 G53 M30 T0000 #Конец программы
\end{minted}

\section*{Приложение А - Операционный эскиз обработки справа}

\begin{sideways}
    \begin{tabular}{ll}
        \textbf{№} & \textbf{(Z, X)}\\
        \hline
        0 & (0, 0)\\
        1 & (15, 70)\\
        2 & (2, 62)\\
        3 & (20, -60)\\
        4 & (0, 32)\\
        5 & (-10, 20)\\
        6 & (-30, 32)\\
        7 & (-60, 20)\\
        8 & (-58, 40)\\
        9 & (15, 0)\\
        10 & (2, 0)\\
        11 & (-30, 0)\\
        12 & (2, 5)\\
        13 & (-20, 5)
    \end{tabular}
\end{sideways}

\begin{figure}[ht]
    \includegraphics[width=.8\linewidth, angle=90]{Figures/sketch1.png}
    \label{fig:sketch2}
\end{figure}

\section*{Приложение Б - Операционный эскиз обработки слева}

\begin{sideways}
    \begin{tabular}{ll}
        \textbf{№} & \textbf{(Z, X)}\\
        \hline
        0 & (0, 0)\\
        1 & (15, 70)\\
        2 & (2, 62)\\
        3 & (-20, 20)\\
        4 & (15, 0)\\
        5 & (2, 0)\\
        6 & (-40, 0)\\
        7 & (2, -5)\\
    \end{tabular}
\end{sideways}

\begin{figure}[ht]
    \includegraphics[width=.8\linewidth, angle=90]{Figures/sketch2.png}
    \label{fig:sketch2}
\end{figure}
