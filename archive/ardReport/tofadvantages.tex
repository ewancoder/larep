Уровень принимаемого сигнала не является надёжным параметром для локализации радиометок внутри помещения. На такое измерение влияет как погрешность во времени, так и факторы конкретной среды (помещения). Погрешность во времени возникает, в основном, из-за добавочного шума и интерференции и может быть значительно понижена за счет усреднения большого количества измерений. Факторы внешней среды --- непредсказуемы и являются случайными. В помещениях с большим количеством препятствий (сложные цеха, склады) измерение расстояния уровнем принимаемого сигнала является неточным.

Метод \textit{Time of Flight} более обещающий подход для измерения расстояния между двумя устройствами. Точное измерение времени должно дать точную оценку расстояния по сравнению с измерением, где используется лишь уровень сигнала. Однако, такой метод требует относительно быстрого и стабильного оборудования.

Благодаря тому, что радиоволны перемещаются со скоростью света и с лёгкостью огибают препятствия (на относительно низкой частоте), Time of Flight является идеальным методом для определения расстояния до объекта, находящегося в сложном помещении (цеха, склады). А измерение точного времени возврата определённого сигнала между двумя объектами позволяет избежать интерференции сигналов (как в случае RSSI).
