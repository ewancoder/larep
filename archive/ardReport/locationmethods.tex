Существует 5 основных методов определения расстояния при помощи радиоволн~\cite{radiofreq:tof}:

\begin{itemize}
    \item Time-Of-Arrival (TOA) и Time-Of-Flight (TOF);
    \item Time-Difference-Of-Arrival (TDOA);
    \item Received-Signal-Strength-Indication (RSSI);
    \item Near-Field-Electromagnetic-Ranging (NREF);
    \item Angle-Of-Arrival (AOA).
\end{itemize}

Метод TOA (TOF) заключается в измерении времени полёта сигнала, на основании которого рассчитывается расстояние.

RSSI в свою очередь измеряет угасание сигнала. Простота этого метода легла в основу широкого спектра устройств.

NREF измеряет сдвиг фазы между электрической и магнитной составляющей, работая на очень низких частотах (порядка 530 -- 1710 кГц).

TDOA использует набор синхроузлов как известные <<радиоточки>>. Гальваническая связь между <<точками>> обязательна для преодоления задержек. Из-за этого TDOA-решения достаточно комплексны и дороги, что ограничивает использование в спектре лишь специфичных архитектурных решений.

AOA использует сложные массивы антенн для измерения угла полученного сигнала. Из-за больших размеров таких массивов данное решение не подходит для компактных систем автоматизации <<радиометками>> в производственных цехах.

В основном, когда рассматривают метод TOF, зачастую сравнивают его с ближайшим <<родственником>> - RSSI. И RSSI, и TOF измеряют расстояние простой посылкой электромагнитного сигнала. Когда RSSI-метод определяет расстояние путём определения, на какую величину уровень сигнала упал по сравнению с исходным уровнем, TOF-метод определяет расстояние путём конкретной отправки сигнала в одну и другую сторону и измерением \textbf{времени} полёта радиоволны на заданном расстоянии.

Отсюда можно сделать очевидный вывод, что RSSI-устройства просты в исполнении, но обладают низкой точностью и интерферируют с подобными устройствами. Когда TOF-устройства достаточно точны, однако требуют высокой частоты работы для высокой точности на маленьких расстояниях.
