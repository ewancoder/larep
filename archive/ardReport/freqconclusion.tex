Учитывая идеальную частоту библиотеки VirualWire в 20 Гц (глава~\ref{sec:virtualwirefreq}), получим сокращение времени ожидания с 4 часов до 12 минут. Однако 20 Гц --- это лишь идеальная частота передачи сообщения. Т. к. сообщение летит в обе стороны, частота уменьшается в 2 раза (10 Гц). Учитывая некоторые задержки и помехи во время передачи, получим частоту менее 10 Гц.

Как было показано в главе~\ref{sec:fs1000a}, частота используемого в данном проекте передатчика достигает 9.6 кГц, что в идеале позволит нам сократить время накопления с 4 часов до двух секунд.

Учитывая, что время накопления в 1 минуту уже будет достаточным условием рациональности проекта, рассчитаем необходимую частоту работы устройства:

$$f = \frac{t_1}{1~\textrm{мин}} = \frac{252~\textrm{мин}}{1~\textrm{мин}} = 252~\textrm{Гц},$$

где $t_1$ --- время накопления с частотой в 1 Гц.

Соответственно, на данном этапе стоит задача создания собственной библиотеки передачи данных, приспособленной для использования на больших частотах, или же поиск более быстрой библиотеки, аналогичной VirtualWire (например, RadioHead).
