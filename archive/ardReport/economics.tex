\subsection{Определение единовременных затрат на разработку модели}

Единовременные капитальные затраты представляют собой цену модели. Различают оптовую и отпускную цены. Все расчеты между покупателем и продавцом продукции, к числу которой относят и модели, производятся на основе отпускных цен. В настоящее время в соответствии с законодательством РБ в отпускную цену наряду с оптовой ценой включается налог на добавленную стоимость.

Определяющим фактором оптовой цены разработки является трудоемкость разработки модели.

\subsubsection{Определение трудоёмкости разработки модели}

Трудоемкость разработки ПП может быть определена на основе Типовых норм времени для программирования задач на ЭВМ. Она включает время на постановку задачи и время на программирование задачи и определяется по формуле~\ref{eq:econcomplexity}:

\begin{equation}
    \label{eq:econcomplexity}
    T_{рз} = (\sum\limits_{i=1}^n T_{посi} + \sum\limits_{i=1}^n T_{прогi}) \cdot 8,
\end{equation}

где n --- количество этапов разработки программы;
$T_{посi}$ --- трудоёмкость постановки задачи на i-м этапе разработки программы, дней;
$T_{прогi}$ --- трудоёмкость программирования задачи на i-м этапе разработки программы, дней.

Нормы времени учитывают ряд факторов, наибольшим образом влияющих на трудоемкость разработки проекта:

\begin{itemize}
    \item количество разновидностей форм входной информации;
    \item количество разновидностей форм выходной информации;
    \item степень новизны задачи;
    \item сложность алгоритма;
    \item вид используемой информации;
    \item сложность контроля входной и выходной информации;
    \item язык программирования;
    \item объем входной информации;
    \item использование типовых решений, типовых проектов и программ, стандартных модулей.
\end{itemize}

Предусмотрено четыре степени новизны разрабатываемых задач:

\begin{enumerate}
    \item Разработка задач, предусматривающая применение принципиально новых методов разработки, проведение научно-исследовательских работ.
    \item Разработка типовых проектных решений, оригинальных задач и систем, не имеющих аналогов.
    \item Разработка проекта с использованием типовых проектных решений при условии их изменения; разработка проектов, имеющих аналогичные решения.
    \item Привязка типовых проектных решений.
\end{enumerate}

Сложность алгоритма представлена тремя группами:

\begin{enumerate}
    \item Алгоритмы оптимизации и моделирования систем и объектов.
    \item Алгоритмы учета, отчетности, статистики и поиска.
    \item Алгоритмы, реализующие стандартные методы решения, а также не предусматривающие применения сложных численных и логических методов.
\end{enumerate}

В ряде случаев (например, если разрабатываемая программа не является законченной системой, а только реализует часть ее функций, или расчет по Типовым нормам времени затруднен) трудоемкость создания такого модель может быть определена укрупненным методом. При этом необходимо воспользоваться формулой~\ref{eq:econtrz}

\begin{equation}
    \label{eq:econtrz}
    \textrm{Т}_\textrm{рз} = \textrm{Т}_\textrm{оа} + \textrm{Т}_\textrm{бс} + \textrm{Т}_\textrm{п} + \textrm{Т}_\textrm{отл} + \textrm{Т}_\textrm{др} + \textrm{Т}_\textrm{до}
\end{equation}
