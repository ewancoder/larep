\begin{minted}[gobble=4,fontsize=\footnotesize,breaklines=true]{cpp}
    // VirtualWire.h
    //
    // Virtual Wire implementation for Arduino
    // See the README file in this directory fdor documentation
    // 
    // Author: Mike McCauley (mikem@airspayce.com) DO NOT CONTACT THE AUTHOR DIRECTLY: USE THE LISTS
    // Copyright (C) 2008 Mike McCauley
    // $Id: VirtualWire.h,v 1.6 2013/02/14 22:02:11 mikem Exp mikem $

    #ifndef VirtualWire_h
    #define VirtualWire_h

    #include <stdlib.h>
    #if defined(ARDUINO)
     #if ARDUINO >= 100
      #include <Arduino.h>
     #else
      #include <wiring.h>
     #endif
    #elif defined(__MSP430G2452__) || defined(__MSP430G2553__) // LaunchPad specific
     #include "legacymsp430.h"
     #include "Energia.h"
    #else // error
     #error Platform not defined
    #endif

    // These defs cause trouble on some versions of Arduino
    #undef abs
    #undef double
    #undef round

    /// Maximum number of bytes in a message, counting the byte count and FCS
    #define VW_MAX_MESSAGE_LEN 30

    /// The maximum payload length
    #define VW_MAX_PAYLOAD VW_MAX_MESSAGE_LEN-3

    /// The size of the receiver ramp. Ramp wraps modulu this number
    #define VW_RX_RAMP_LEN 160

    /// Number of samples per bit
    #define VW_RX_SAMPLES_PER_BIT 8

    // Ramp adjustment parameters
    // Standard is if a transition occurs before VW_RAMP_TRANSITION (80) in the ramp,
    // the ramp is retarded by adding VW_RAMP_INC_RETARD (11)
    // else by adding VW_RAMP_INC_ADVANCE (29)
    // If there is no transition it is adjusted by VW_RAMP_INC (20)
    /// Internal ramp adjustment parameter
    #define VW_RAMP_INC (VW_RX_RAMP_LEN/VW_RX_SAMPLES_PER_BIT)
    /// Internal ramp adjustment parameter
    #define VW_RAMP_TRANSITION VW_RX_RAMP_LEN/2
    /// Internal ramp adjustment parameter
    #define VW_RAMP_ADJUST 9
    /// Internal ramp adjustment parameter
    #define VW_RAMP_INC_RETARD (VW_RAMP_INC-VW_RAMP_ADJUST)
    /// Internal ramp adjustment parameter
    #define VW_RAMP_INC_ADVANCE (VW_RAMP_INC+VW_RAMP_ADJUST)

    /// Outgoing message bits grouped as 6-bit words
    /// 36 alternating 1/0 bits, followed by 12 bits of start symbol
    /// Followed immediately by the 4-6 bit encoded byte count, 
    /// message buffer and 2 byte FCS
    /// Each byte from the byte count on is translated into 2x6-bit words
    /// Caution, each symbol is transmitted LSBit first, 
    /// but each byte is transmitted high nybble first
    #define VW_HEADER_LEN 8

    // Cant really do this as a real C++ class, since we need to have 
    // an ISR
    extern "C"
    {
        /// Set the digital IO pin to be for transmit data. 
        /// This pin will only be accessed if
        /// the transmitter is enabled
        /// \param[in] pin The Arduino pin number for transmitting data. Defaults to 12.
        extern void vw_set_tx_pin(uint8_t pin);

        /// Set the digital IO pin to be for receive data.
        /// This pin will only be accessed if
        /// the receiver is enabled
        /// \param[in] pin The Arduino pin number for receiving data. Defaults to 11.
        extern void vw_set_rx_pin(uint8_t pin);

        // Set the digital IO pin to enable the transmitter (press to talk, PTT)'
        /// This pin will only be accessed if
        /// the transmitter is enabled
        /// \param[in] pin The Arduino pin number to enable the transmitter. Defaults to 10.
        extern void vw_set_ptt_pin(uint8_t pin);

        /// By default the PTT pin goes high when the transmitter is enabled.
        /// This flag forces it low when the transmitter is enabled.
        /// \param[in] inverted True to invert PTT
        extern void vw_set_ptt_inverted(uint8_t inverted);

        /// Initialise the VirtualWire software, to operate at speed bits per second
        /// Call this one in your setup() after any vw_set_* calls
        /// Must call vw_rx_start() before you will get any messages
        /// \param[in] speed Desired speed in bits per second
        extern void vw_setup(uint16_t speed);

        /// Start the Phase Locked Loop listening to the receiver
        /// Must do this before you can receive any messages
        /// When a message is available (good checksum or not), vw_have_message();
        /// will return true.
        extern void vw_rx_start();

        /// Stop the Phase Locked Loop listening to the receiver
        /// No messages will be received until vw_rx_start() is called again
        /// Saves interrupt processing cycles
        extern void vw_rx_stop();

        /// Returns the state of the
        /// transmitter
        /// \return true if the transmitter is active else false
        extern uint8_t vx_tx_active();

        /// Block until the transmitter is idle
        /// then returns
        extern void vw_wait_tx();

        /// Block until a message is available
        /// then returns
        extern void vw_wait_rx();

        /// Block until a message is available or for a max time
        /// \param[in] milliseconds Maximum time to wait in milliseconds.
        /// \return true if a message is available, false if the wait timed out.
        extern uint8_t vw_wait_rx_max(unsigned long milliseconds);

        /// Send a message with the given length. Returns almost immediately,
        /// and message will be sent at the right timing by interrupts
        /// \param[in] buf Pointer to the data to transmit
        /// \param[in] len Number of octetes to transmit
        /// \return true if the message was accepted for transmission, false if the message is too long (>VW_MAX_MESSAGE_LEN - 3)
        extern uint8_t vw_send(uint8_t* buf, uint8_t len);

        // Returns true if an unread message is available
        /// \return true if a message is available to read
        extern uint8_t vw_have_message();

        // If a message is available (good checksum or not), copies
        // up to *len octets to buf.
        /// \param[in] buf Pointer to location to save the read data (must be at least *len bytes.
        /// \param[in,out] len Available space in buf. Will be set to the actual number of octets read
        /// \return true if there was a message and the checksum was good
        extern uint8_t vw_get_message(uint8_t* buf, uint8_t* len);
    }

    /// @example client.pde
    /// Client side of simple client/server pair using VirtualWire

    /// @example server.pde
    /// Server side of simple client/server pair using VirtualWire

    /// @example transmitter.pde
    /// Transmitter side of simple one-way transmitter->receiver pair using VirtualWire

    /// @example receiver.pde
    /// Transmitter side of simple one-way transmitter->receiver pair using VirtualWire

    #endif
\end{minted}

\begin{minted}[gobble=4,fontsize=\footnotesize,breaklines=true]{cpp}
    // VirtualWire.cpp
    //
    // Virtual Wire implementation for Arduino
    // See the README file in this directory fdor documentation
    // See also
    // ASH Transceiver Software Designer's Guide of 2002.08.07
    //   http://www.rfm.com/products/apnotes/tr_swg05.pdf
    //
    // Changes:
    // 1.5 2008-05-25: fixed a bug that could prevent messages with certain
    //  bytes sequences being received (false message start detected)
    // 1.6 2011-09-10: Patch from David Bath to prevent unconditional reenabling of the receiver
    //  at end of transmission.
    //
    // Author: Mike McCauley (mikem@airspayce.com)
    // Copyright (C) 2008 Mike McCauley
    // $Id: VirtualWire.cpp,v 1.9 2013/02/14 22:02:11 mikem Exp mikem $


    #if defined(ARDUINO)
     #if (ARDUINO < 100)
      #include "WProgram.h"
     #endif
    #elif defined(__MSP430G2452__) || defined(__MSP430G2553__) // LaunchPad specific
     #include "legacymsp430.h"
     #include "Energia.h"
    #else // error
     #error Platform not defined
    #endif

    #include "VirtualWire.h"
    #include <util/crc16.h>


    static uint8_t vw_tx_buf[(VW_MAX_MESSAGE_LEN * 2) + VW_HEADER_LEN] 
         = {0x2a, 0x2a, 0x2a, 0x2a, 0x2a, 0x2a, 0x38, 0x2c};

    // Number of symbols in vw_tx_buf to be sent;
    static uint8_t vw_tx_len = 0;

    // Index of the next symbol to send. Ranges from 0 to vw_tx_len
    static uint8_t vw_tx_index = 0;

    // Bit number of next bit to send
    static uint8_t vw_tx_bit = 0;

    // Sample number for the transmitter. Runs 0 to 7 during one bit interval
    static uint8_t vw_tx_sample = 0;

    // Flag to indicated the transmitter is active
    static volatile uint8_t vw_tx_enabled = 0;

    // Total number of messages sent
    static uint16_t vw_tx_msg_count = 0;

    // The digital IO pin number of the press to talk, enables the transmitter hardware
    static uint8_t vw_ptt_pin = 10;
    static uint8_t vw_ptt_inverted = 0;

    // The digital IO pin number of the receiver data
    static uint8_t vw_rx_pin = 11;

    // The digital IO pin number of the transmitter data
    static uint8_t vw_tx_pin = 12;

    // Current receiver sample
    static uint8_t vw_rx_sample = 0;

    // Last receiver sample
    static uint8_t vw_rx_last_sample = 0;

    // PLL ramp, varies between 0 and VW_RX_RAMP_LEN-1 (159) over 
    // VW_RX_SAMPLES_PER_BIT (8) samples per nominal bit time. 
    // When the PLL is synchronised, bit transitions happen at about the
    // 0 mark. 
    static uint8_t vw_rx_pll_ramp = 0;

    // This is the integrate and dump integral. If there are <5 0 samples in the PLL cycle
    // the bit is declared a 0, else a 1
    static uint8_t vw_rx_integrator = 0;

    // Flag indictate if we have seen the start symbol of a new message and are
    // in the processes of reading and decoding it
    static uint8_t vw_rx_active = 0;

    // Flag to indicate that a new message is available
    static volatile uint8_t vw_rx_done = 0;

    // Flag to indicate the receiver PLL is to run
    static uint8_t vw_rx_enabled = 0;

    // Last 12 bits received, so we can look for the start symbol
    static uint16_t vw_rx_bits = 0;

    // How many bits of message we have received. Ranges from 0 to 12
    static uint8_t vw_rx_bit_count = 0;

    // The incoming message buffer
    static uint8_t vw_rx_buf[VW_MAX_MESSAGE_LEN];

    // The incoming message expected length
    static uint8_t vw_rx_count = 0;

    // The incoming message buffer length received so far
    static volatile uint8_t vw_rx_len = 0;

    // Number of bad messages received and dropped due to bad lengths
    static uint8_t vw_rx_bad = 0;

    // Number of good messages received
    static uint8_t vw_rx_good = 0;

    // 4 bit to 6 bit symbol converter table
    // Used to convert the high and low nybbles of the transmitted data
    // into 6 bit symbols for transmission. Each 6-bit symbol has 3 1s and 3 0s 
    // with at most 3 consecutive identical bits
    static uint8_t symbols[] =
    {
        0xd,  0xe,  0x13, 0x15, 0x16, 0x19, 0x1a, 0x1c, 
        0x23, 0x25, 0x26, 0x29, 0x2a, 0x2c, 0x32, 0x34
    };

    // Cant really do this as a real C++ class, since we need to have 
    // an ISR
    extern "C"
    {

    // Compute CRC over count bytes.
    // This should only be ever called at user level, not interrupt level
    uint16_t vw_crc(uint8_t *ptr, uint8_t count)
    {
        uint16_t crc = 0xffff;

        while (count-- > 0) 
        crc = _crc_ccitt_update(crc, *ptr++);
        return crc;
    }

    // Convert a 6 bit encoded symbol into its 4 bit decoded equivalent
    uint8_t vw_symbol_6to4(uint8_t symbol)
    {
        uint8_t i;
        
        // Linear search :-( Could have a 64 byte reverse lookup table?
        for (i = 0; i < 16; i++)
        if (symbol == symbols[i]) return i;
        return 0; // Not found
    }

    // Set the output pin number for transmitter data
    void vw_set_tx_pin(uint8_t pin)
    {
        vw_tx_pin = pin;
    }

    // Set the pin number for input receiver data
    void vw_set_rx_pin(uint8_t pin)
    {
        vw_rx_pin = pin;
    }

    // Set the output pin number for transmitter PTT enable
    void vw_set_ptt_pin(uint8_t pin)
    {
        vw_ptt_pin = pin;
    }

    // Set the ptt pin inverted (low to transmit)
    void vw_set_ptt_inverted(uint8_t inverted)
    {
        vw_ptt_inverted = inverted;
    }

    // Called 8 times per bit period
    // Phase locked loop tries to synchronise with the transmitter so that bit 
    // transitions occur at about the time vw_rx_pll_ramp is 0;
    // Then the average is computed over each bit period to deduce the bit value
    void vw_pll()
    {
        // Integrate each sample
        if (vw_rx_sample)
        vw_rx_integrator++;

        if (vw_rx_sample != vw_rx_last_sample)
        {
        // Transition, advance if ramp > 80, retard if < 80
        vw_rx_pll_ramp += ((vw_rx_pll_ramp < VW_RAMP_TRANSITION) 
                   ? VW_RAMP_INC_RETARD 
                   : VW_RAMP_INC_ADVANCE);
        vw_rx_last_sample = vw_rx_sample;
        }
        else
        {
        // No transition
        // Advance ramp by standard 20 (== 160/8 samples)
        vw_rx_pll_ramp += VW_RAMP_INC;
        }
        if (vw_rx_pll_ramp >= VW_RX_RAMP_LEN)
        {
        // Add this to the 12th bit of vw_rx_bits, LSB first
        // The last 12 bits are kept
        vw_rx_bits >>= 1;

        // Check the integrator to see how many samples in this cycle were high.
        // If < 5 out of 8, then its declared a 0 bit, else a 1;
        if (vw_rx_integrator >= 5)
            vw_rx_bits |= 0x800;

        vw_rx_pll_ramp -= VW_RX_RAMP_LEN;
        vw_rx_integrator = 0; // Clear the integral for the next cycle

        if (vw_rx_active)
        {
            // We have the start symbol and now we are collecting message bits,
            // 6 per symbol, each which has to be decoded to 4 bits
            if (++vw_rx_bit_count >= 12)
            {
            // Have 12 bits of encoded message == 1 byte encoded
            // Decode as 2 lots of 6 bits into 2 lots of 4 bits
            // The 6 lsbits are the high nybble
            uint8_t this_byte = 
                (vw_symbol_6to4(vw_rx_bits & 0x3f)) << 4 
                | vw_symbol_6to4(vw_rx_bits >> 6);

            // The first decoded byte is the byte count of the following message
            // the count includes the byte count and the 2 trailing FCS bytes
            // REVISIT: may also include the ACK flag at 0x40
            if (vw_rx_len == 0)
            {
                // The first byte is the byte count
                // Check it for sensibility. It cant be less than 4, since it
                // includes the bytes count itself and the 2 byte FCS
                vw_rx_count = this_byte;
                if (vw_rx_count < 4 || vw_rx_count > VW_MAX_MESSAGE_LEN)
                {
                // Stupid message length, drop the whole thing
                vw_rx_active = false;
                vw_rx_bad++;
                            return;
                }
            }
            vw_rx_buf[vw_rx_len++] = this_byte;

            if (vw_rx_len >= vw_rx_count)
            {
                // Got all the bytes now
                vw_rx_active = false;
                vw_rx_good++;
                vw_rx_done = true; // Better come get it before the next one starts
            }
            vw_rx_bit_count = 0;
            }
        }
        // Not in a message, see if we have a start symbol
        else if (vw_rx_bits == 0xb38)
        {
            // Have start symbol, start collecting message
            vw_rx_active = true;
            vw_rx_bit_count = 0;
            vw_rx_len = 0;
            vw_rx_done = false; // Too bad if you missed the last message
        }
        }
    }

    // Common function for setting timer ticks @ prescaler values for speed
    // Returns prescaler index into {0, 1, 8, 64, 256, 1024} array
    // and sets nticks to compare-match value if lower than max_ticks
    // returns 0 & nticks = 0 on fault
    static uint8_t _timer_calc(uint16_t speed, uint16_t max_ticks, uint16_t *nticks)
    {
        // Clock divider (prescaler) values - 0/3333: error flag
        uint16_t prescalers[] = {0, 1, 8, 64, 256, 1024, 3333};
        uint8_t prescaler=0; // index into array & return bit value
        unsigned long ulticks; // calculate by ntick overflow

        // Div-by-zero protection
        if (speed == 0)
        {
            // signal fault
            *nticks = 0;
            return 0;
        }

        // test increasing prescaler (divisor), decreasing ulticks until no overflow
        for (prescaler=1; prescaler < 7; prescaler += 1)
        {
            // Amount of time per CPU clock tick (in seconds)
            float clock_time = (1.0 / (float(F_CPU) / float(prescalers[prescaler])));
            // Fraction of second needed to xmit one bit
            float bit_time = ((1.0 / float(speed)) / 8.0);
            // number of prescaled ticks needed to handle bit time @ speed
            ulticks = long(bit_time / clock_time);
            // Test if ulticks fits in nticks bitwidth (with 1-tick safety margin)
            if ((ulticks > 1) && (ulticks < max_ticks))
            {
                break; // found prescaler
            }
            // Won't fit, check with next prescaler value
        }

        // Check for error
        if ((prescaler == 6) || (ulticks < 2) || (ulticks > max_ticks))
        {
            // signal fault
            *nticks = 0;
            return 0;
        }

        *nticks = ulticks;
        return prescaler;
    }

    #if defined(__arm__) && defined(CORE_TEENSY)
      // This allows the AVR interrupt code below to be run from an
      // IntervalTimer object.  It must be above vw_setup(), so the
      // the TIMER1_COMPA_vect function name is defined.
      #ifdef SIGNAL
      #undef SIGNAL
      #endif
      #define SIGNAL(f) void f(void)
      #ifdef TIMER1_COMPA_vect
      #undef TIMER1_COMPA_vect
      #endif
      void TIMER1_COMPA_vect(void);
    #endif


    // Speed is in bits per sec RF rate
    #if defined(__MSP430G2452__) || defined(__MSP430G2553__) // LaunchPad specific
    void vw_setup(uint16_t speed)
    {
        // Calculate the counter overflow count based on the required bit speed
        // and CPU clock rate
        uint16_t ocr1a = (F_CPU / 8UL) / speed;
            
        // This code is for Energia/MSP430
        TA0CCR0 = ocr1a;				// Ticks for 62,5 us
        TA0CTL = TASSEL_2 + MC_1;       // SMCLK, up mode
        TA0CCTL0 |= CCIE;               // CCR0 interrupt enabled
            
        // Set up digital IO pins
        pinMode(vw_tx_pin, OUTPUT);
        pinMode(vw_rx_pin, INPUT);
        pinMode(vw_ptt_pin, OUTPUT);
        digitalWrite(vw_ptt_pin, vw_ptt_inverted);
    }	

    #elif defined (ARDUINO) // Arduino specific
    void vw_setup(uint16_t speed)
    {
        uint16_t nticks; // number of prescaled ticks needed
        uint8_t prescaler; // Bit values for CS0[2:0]

    #ifdef __AVR_ATtiny85__
        // figure out prescaler value and counter match value
        prescaler = _timer_calc(speed, (uint8_t)-1, &nticks);
        if (!prescaler)
        {
            return; // fault
        }

        TCCR0A = 0;
        TCCR0A = _BV(WGM01); // Turn on CTC mode / Output Compare pins disconnected

        // convert prescaler index to TCCRnB prescaler bits CS00, CS01, CS02
        TCCR0B = 0;
        TCCR0B = prescaler; // set CS00, CS01, CS02 (other bits not needed)

        // Number of ticks to count before firing interrupt
        OCR0A = uint8_t(nticks);

        // Set mask to fire interrupt when OCF0A bit is set in TIFR0
        TIMSK |= _BV(OCIE0A);

    #elif defined(__arm__) && defined(CORE_TEENSY)
        // on Teensy 3.0 (32 bit ARM), use an interval timer
        IntervalTimer *t = new IntervalTimer();
        t->begin(TIMER1_COMPA_vect, 125000.0 / (float)(speed));

    #else // ARDUINO
        // This is the path for most Arduinos
        // figure out prescaler value and counter match value
        prescaler = _timer_calc(speed, (uint16_t)-1, &nticks);
        if (!prescaler)
        {
            return; // fault
        }

        TCCR1A = 0; // Output Compare pins disconnected
        TCCR1B = _BV(WGM12); // Turn on CTC mode

        // convert prescaler index to TCCRnB prescaler bits CS10, CS11, CS12
        TCCR1B |= prescaler;

        // Caution: special procedures for setting 16 bit regs
        // is handled by the compiler
        OCR1A = nticks;
        // Enable interrupt
    #ifdef TIMSK1
        // atmega168
        TIMSK1 |= _BV(OCIE1A);
    #else
        // others
        TIMSK |= _BV(OCIE1A);
    #endif // TIMSK1

    #endif // __AVR_ATtiny85__

        // Set up digital IO pins
        pinMode(vw_tx_pin, OUTPUT);
        pinMode(vw_rx_pin, INPUT);
        pinMode(vw_ptt_pin, OUTPUT);
        digitalWrite(vw_ptt_pin, vw_ptt_inverted);
    }

    #endif // ARDUINO

    // Start the transmitter, call when the tx buffer is ready to go and vw_tx_len is
    // set to the total number of symbols to send
    void vw_tx_start()
    {
        vw_tx_index = 0;
        vw_tx_bit = 0;
        vw_tx_sample = 0;

        // Enable the transmitter hardware
        digitalWrite(vw_ptt_pin, true ^ vw_ptt_inverted);

        // Next tick interrupt will send the first bit
        vw_tx_enabled = true;
    }

    // Stop the transmitter, call when all bits are sent
    void vw_tx_stop()
    {
        // Disable the transmitter hardware
        digitalWrite(vw_ptt_pin, false ^ vw_ptt_inverted);
        digitalWrite(vw_tx_pin, false);

        // No more ticks for the transmitter
        vw_tx_enabled = false;
    }

    // Enable the receiver. When a message becomes available, vw_rx_done flag
    // is set, and vw_wait_rx() will return.
    void vw_rx_start()
    {
        if (!vw_rx_enabled)
        {
        vw_rx_enabled = true;
        vw_rx_active = false; // Never restart a partial message
        }
    }

    // Disable the receiver
    void vw_rx_stop()
    {
        vw_rx_enabled = false;
    }

    // Return true if the transmitter is active
    uint8_t vx_tx_active()
    {
        return vw_tx_enabled;
    }

    // Wait for the transmitter to become available
    // Busy-wait loop until the ISR says the message has been sent
    void vw_wait_tx()
    {
        while (vw_tx_enabled)
        ;
    }

    // Wait for the receiver to get a message
    // Busy-wait loop until the ISR says a message is available
    // can then call vw_get_message()
    void vw_wait_rx()
    {
        while (!vw_rx_done)
        ;
    }

    // Wait at most max milliseconds for the receiver to receive a message
    // Return the truth of whether there is a message
    uint8_t vw_wait_rx_max(unsigned long milliseconds)
    {
        unsigned long start = millis();

        while (!vw_rx_done && ((millis() - start) < milliseconds))
        ;
        return vw_rx_done;
    }

    // Wait until transmitter is available and encode and queue the message
    // into vw_tx_buf
    // The message is raw bytes, with no packet structure imposed
    // It is transmitted preceded a byte count and followed by 2 FCS bytes
    uint8_t vw_send(uint8_t* buf, uint8_t len)
    {
        uint8_t i;
        uint8_t index = 0;
        uint16_t crc = 0xffff;
        uint8_t *p = vw_tx_buf + VW_HEADER_LEN; // start of the message area
        uint8_t count = len + 3; // Added byte count and FCS to get total number of bytes

        if (len > VW_MAX_PAYLOAD)
        return false;

        // Wait for transmitter to become available
        vw_wait_tx();

        // Encode the message length
        crc = _crc_ccitt_update(crc, count);
        p[index++] = symbols[count >> 4];
        p[index++] = symbols[count & 0xf];

        // Encode the message into 6 bit symbols. Each byte is converted into 
        // 2 6-bit symbols, high nybble first, low nybble second
        for (i = 0; i < len; i++)
        {
        crc = _crc_ccitt_update(crc, buf[i]);
        p[index++] = symbols[buf[i] >> 4];
        p[index++] = symbols[buf[i] & 0xf];
        }

        // Append the fcs, 16 bits before encoding (4 6-bit symbols after encoding)
        // Caution: VW expects the _ones_complement_ of the CCITT CRC-16 as the FCS
        // VW sends FCS as low byte then hi byte
        crc = ~crc;
        p[index++] = symbols[(crc >> 4)  & 0xf];
        p[index++] = symbols[crc & 0xf];
        p[index++] = symbols[(crc >> 12) & 0xf];
        p[index++] = symbols[(crc >> 8)  & 0xf];

        // Total number of 6-bit symbols to send
        vw_tx_len = index + VW_HEADER_LEN;

        // Start the low level interrupt handler sending symbols
        vw_tx_start();

        return true;
    }

    // Return true if there is a message available
    uint8_t vw_have_message()
    {
        return vw_rx_done;
    }

    // Get the last message received (without byte count or FCS)
    // Copy at most *len bytes, set *len to the actual number copied
    // Return true if there is a message and the FCS is OK
    uint8_t vw_get_message(uint8_t* buf, uint8_t* len)
    {
        uint8_t rxlen;
        
        // Message available?
        if (!vw_rx_done)
        return false;
        
        // Wait until vw_rx_done is set before reading vw_rx_len
        // then remove bytecount and FCS
        rxlen = vw_rx_len - 3;
        
        // Copy message (good or bad)
        if (*len > rxlen)
        *len = rxlen;
        memcpy(buf, vw_rx_buf + 1, *len);
        
        vw_rx_done = false; // OK, got that message thanks
        
        // Check the FCS, return goodness
        return (vw_crc(vw_rx_buf, vw_rx_len) == 0xf0b8); // FCS OK?
    }

    // This is the interrupt service routine called when timer1 overflows
    // Its job is to output the next bit from the transmitter (every 8 calls)
    // and to call the PLL code if the receiver is enabled
    //ISR(SIG_OUTPUT_COMPARE1A)
    #if defined (ARDUINO) // Arduino specific

    #ifdef __AVR_ATtiny85__
    SIGNAL(TIM0_COMPA_vect)
    #else // Assume Arduino Uno (328p or similar)

    SIGNAL(TIMER1_COMPA_vect)
    #endif // __AVR_ATtiny85__

    {
        if (vw_rx_enabled && !vw_tx_enabled)
        vw_rx_sample = digitalRead(vw_rx_pin);
        
        // Do transmitter stuff first to reduce transmitter bit jitter due 
        // to variable receiver processing
        if (vw_tx_enabled && vw_tx_sample++ == 0)
        {
        // Send next bit
        // Symbols are sent LSB first
        // Finished sending the whole message? (after waiting one bit period 
        // since the last bit)
        if (vw_tx_index >= vw_tx_len)
        {
            vw_tx_stop();
            vw_tx_msg_count++;
        }
        else
        {
            digitalWrite(vw_tx_pin, vw_tx_buf[vw_tx_index] & (1 << vw_tx_bit++));
            if (vw_tx_bit >= 6)
            {
            vw_tx_bit = 0;
            vw_tx_index++;
            }
        }
        }
        if (vw_tx_sample > 7)
        vw_tx_sample = 0;
        
        if (vw_rx_enabled && !vw_tx_enabled)
        vw_pll();
    }
    #elif defined(__MSP430G2452__) || defined(__MSP430G2553__) // LaunchPad specific
    void vw_Int_Handler()
    {
        if (vw_rx_enabled && !vw_tx_enabled)
        vw_rx_sample = digitalRead(vw_rx_pin);
        
        // Do transmitter stuff first to reduce transmitter bit jitter due 
        // to variable receiver processing
        if (vw_tx_enabled && vw_tx_sample++ == 0)
        {
        // Send next bit
        // Symbols are sent LSB first
        // Finished sending the whole message? (after waiting one bit period 
        // since the last bit)
        if (vw_tx_index >= vw_tx_len)
        {
            vw_tx_stop();
            vw_tx_msg_count++;
        }
        else
        {
            digitalWrite(vw_tx_pin, vw_tx_buf[vw_tx_index] & (1 << vw_tx_bit++));
            if (vw_tx_bit >= 6)
            {
            vw_tx_bit = 0;
            vw_tx_index++;
            }
        }
        }
        if (vw_tx_sample > 7)
        vw_tx_sample = 0;
        
        if (vw_rx_enabled && !vw_tx_enabled)
        vw_pll();
    }

    interrupt(TIMER0_A0_VECTOR) Timer_A_int(void) 
    {
        vw_Int_Handler();
    };

    #endif


    }
\end{minted}

\begin{minted}[gobble=4,fontsize=\footnotesize, breaklines=true]{cpp}
    // util/crc16.h
    #ifndef _UTIL_CRC16_H_
    #define _UTIL_CRC16_H_

    #include <stdint.h>

    #define lo8(x) ((x)&0xff) 
    #define hi8(x) ((x)>>8)

        uint16_t crc16_update(uint16_t crc, uint8_t a)
        {
        int i;

        crc ^= a;
        for (i = 0; i < 8; ++i)
        {
            if (crc & 1)
            crc = (crc >> 1) ^ 0xA001;
            else
            crc = (crc >> 1);
        }

        return crc;
        }

        uint16_t crc_xmodem_update (uint16_t crc, uint8_t data)
        {
            int i;

            crc = crc ^ ((uint16_t)data << 8);
            for (i=0; i<8; i++)
            {
                if (crc & 0x8000)
                    crc = (crc << 1) ^ 0x1021;
                else
                    crc <<= 1;
            }

            return crc;
        }
        uint16_t _crc_ccitt_update (uint16_t crc, uint8_t data)
        {
            data ^= lo8 (crc);
            data ^= data << 4;

            return ((((uint16_t)data << 8) | hi8 (crc)) ^ (uint8_t)(data >> 4) 
                    ^ ((uint16_t)data << 3));
        }

        uint8_t _crc_ibutton_update(uint8_t crc, uint8_t data)
        {
        uint8_t i;

        crc = crc ^ data;
        for (i = 0; i < 8; i++)
        {
            if (crc & 0x01)
                crc = (crc >> 1) ^ 0x8C;
            else
                crc >>= 1;
        }

        return crc;
        }


    #endif /* _UTIL_CRC16_H_ */
\end{minted}
