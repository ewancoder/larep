Программа-приёмник осуществляет приём и передачу сигнала назад, в то же время рассчитывая время Time of Flight, используя локальные часы (Two Way Time Transfer, глава~\ref{sec:mitigation}). Приёмник принимает сигнал в момент времени $t_{Rr}$ и сразу же пересылает его назад, при этом сохраняя результаты в памяти для последующих рассчётов. Время, которое приёмник ожидает поступления следующего сигнала и является временем \textit{Time of Flight}, умноженным на два.

Структура программы-приёмника представлена ниже:

\begin{minted}[gobble=4,fontsize=\footnotesize]{c}
    void setup()
    {
        Serial.begin(115200);
        // Start transmitting.
    }

    void loop()
    {
        ReadData(100); // 100 msec timeout.
        if (Success)
        {
            SendData("1");
            // Calculate ToF.
        }
        else
        {
            // Mitigate.
        }
    }
\end{minted}

Учитывая ограничения VirtualWire (как было показано в главе~\ref{sec:virtualwirefreq}), а также ограничения используемых датчиков (глава~\ref{sec:fs1000a}), построить работающий прототип с её использованием не представляется возможным. Библиотека VirtualWire была использована для \textbf{моделирования} части, отвечающей за беспроводную передачу пакета и рассчёт времени Time of Flight.

Ниже представлен исходный код программы <<общения>> клиента с сервером без расчёта времени TOF. Данный код был исполнен на программной среде Arduino в целях моделирования связи между устройствами.

\begin{minted}[gobble=4,fontsize=\footnotesize]{c}
    #include <VirtualWire.h>
    int tx = 1;
    int rx = 0;

    void setup()
    {
        Serial.begin(115200);
        vw_set_tx_pin(tx);
        vw_set_rx_pin(rx);
        vw_set_ptt_inverted(true);
        vw_setup(9600);
        vw_rx_start();
    }

    void loop()
    {
        const char *msg = "1"; // Send back 1.
        uint8_t buf[VW_MAX_MESSAGE_LEN];
        uint8_t buflen = VW_MAX_MESSAGE_LEN;

        vw_wait_rx();
        if (vw_get_message(buf, &buflen))
        {
            vw_send((uint8_t *)msg, strlen(msg));
        }
    }
\end{minted}
