Быстрое развитие промышленной автоматизации обусловило развитие методов и средств сбора данных и управления, начиная с всевозможных датчиков и заканчивая SCADA-системами. С развитием SCADA-систем возникла необходимость учёта и контроля многих узлов системы, не привязанных к измерительному оборудованию. Для обхода данного препятствия были задействованы беспроводные технологии: Wi-Fi, RFID, датчики определения местоположения объекта. В работе рассматривается внутренняя структура и алгоритм работы устройства определения расстояния между объектами методом \textit{Time of Flight}.

Радиолокация --- область науки и техники, объединяющая методы и средства локации (обнаружения и измерения координат) и определения свойств различных объектов с помощью радиоволн~\cite{wiki:radiolocation}.

Радиолокация позволяет использовать радиоволны для автоматизации различных целей и задач, в том числе --- автоматизации производств, АСУТП-систем и SCADA. Например, используя специальные <<радиометки>>, закрепляемые на движущихся объектах, можно отслеживать пути перемещения различного технологического оборудования, заготовок на складах и даже передвижение персонала.

В то время, как GPS, ГЛОНАСС и подобные системы позволяют отслеживать перемещение объекта на открытой местности, они не позволяют отслеживать его внутри помещений. Всвязи с этим, в современных условиях промышленной автоматизации стоит задача создания систем отслеживания местоположения всех критических объектов внутри цехов.

Простейшая задача радиолокации сводится к вычислению расстояний между тремя точками, а затем вычислению местоположения специализированными алгоритмами (например, методом триангуляции). Дипломная работа сфокусирована на определении расстояния между двумя точками методом \textit{Time of Flight}, а также расчётё погрешностей и задержек, возникающих во время измерения, и, как следствие, разрешающей возможности конечного устройства в зависимости от выбранного оборудования.
