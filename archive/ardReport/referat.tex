Дипломный проект: 105 страниц, 33 рисунка, 27 таблиц, 18 источников, 3 приложения.

ИЗМЕРЕНИЕ РАССТОЯНИЯ, МЕТОД TIME OF FLIGHT, ВСТРАИВАЕМЫЙ ПРОГРАММНЫЙ МОДУЛЬ

Объектом исследования является система отслеживания местоположения подвижных объектов на производстве (складе).

Цель проекта --- разработать устройство определения расстояния радиочастотным методом <<Time of Flight>>, вычислить возникающие в процессе определения расстояния задержки и погрешности, определить их природу, подобрать наиболее подходящее оборудование.

В процессе работы проводился теоретический анализ процессов, возникающих при пересылке сообщения по радиоканалу, сравнение существующих методов определения расстояния, экспериментальное моделирование различных частей проекта, определение возникающих при этом задержек.

В результате исследования была разработана аппаратно-программная модель измерения времени полёта ToF.

Областью практического применения результатов дипломного проектирования является автоматизация сбора данных о местоположении различных объектах на складах или подвижных механизмов на производстве.

Студент-дипломник подтверждает, что приведённый в дипломном проекте расчётно-аналитический материал объективно отражает состояние исследуемого процесса, все заимствованные из литературных и других источников теоретические и методологические положения и концепции сопровождаются ссылками на их авторов.
