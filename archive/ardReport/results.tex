В результате моделирования были получены нерелевантные данные, т. к. частота работы имеющегося оборудования гораздо ниже необходимых (глава~\ref{sec:taskfreq}).

Как было показано в главе~\ref{sec:mitigation}, для целей измерения Time of Flight выгоднее использовать более высокочастотные, микропроцессорные технологии, работающие на гигагерцовых частотах. Также, Wi-Fi-диапазон частот (2.4ГГц) для этой цели подойдёт больше, чем свободная полоса 433 МГц.

Для повышения надёжности всей системы, необходимо увеличить размер передаваемого сообщения, что сказывается на требованиях к скоростям радиодатчиков.

Получившиеся в главе~\ref{sec:taskfreq} задержки можно вычислить аналитически, используя данные платформы Arduino/ATMega (процессора ATMega 328) о количестве тактов, необходимых для проведения определённых операций. Библиотека WirtualWire (приложение В) содержит множество команд, CRC-проверку и формирование достаточно большго пакета бит для стабильной отправки сообщения, поэтому в ней возникают наибольшие задержки.

Общие задержки складываются из трёх факторов:

\begin{enumerate}
    \item Инициализация радиопередатчика и приём сообщения радиоприёмником.
    \item Передача электрических импульсов на входы и выходы микроконтроллера, задержки в портах, наводки.
    \item Задержки, связанные с выполнением инструкций микропроцессором.
\end{enumerate}

Первая группа задержек определяется характеристиками приёмника и передатчика. Так как в проекте был использован дешёвый передатчик FS1000A со скоростью передачи данных не более 9600 Кбит/с, задержки на инициализацию радиопередатчика и приём сообщения могут быть достаточно большими чтоб учитывать их в расчёте общей задержки. Также, на скорость и качество радиопередачи будет влиять длина и правильность антенны.

На вторую группу задержек влияют такие факторы, как наличие белого шума, расстояние между контактами и т. д. Вторая группа, как правило, наименьшая из трёх. При поверхностной оценке задержек эту группу можно опустить, однако для построения точного Time of Flight устройства с наилучшими параметрами работы необходимо учесть все данные при проектировании печатной платы.

Третья группа задержек является основной, минимизировать которую представляется возможным путём оптимизации программного модуля. При использовании ARM-ядра, количество инструкций можно сократить. Когда умножение двух чисел в AVR-ядре займёт 5 инструкций, в ARM-ядре оно будет вычислено в одну инструкцию. Проведя анализ программного кода, можно рассчитать общую задержку, связанную с выполнением инструкций в AVR-ядре. После этого, можно получить задержку, связанную с радиопередачей сообщения, используя экспериментальные данные об общей задержке всей передачи.
