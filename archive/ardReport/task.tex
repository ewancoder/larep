Метод \textit{Time of Flight} заключается в измерении времени полёта электромагнитной волны от источника к приёмнику и обратно. На основе этого осуществляется расчёт расстояния между двумя точками. \textit{Time of Flight} применяется внутри помещений с большим количеством <<радиометок>>, где необходима большая точность. Однако, с точностью растут и необходимые требования к конечному устройству.

Всвязи с этим, ставятся следующие задачи:

\begin{itemize}
    \item устройство должно работать на относительно высокой стабильной частоте: для точности измерения в 1 метр частота работы должна быть 300 МГц;
    \item необходима точная синхронизация времени: расхождение начального отсчёта времени источника и приёмника значительно скажется на результате измерения;
    \item мельчайшие шумы будут сказываться на результате, уменьшая точность измерения;
    \item необходимо учитывать всевозможные задержки оборудования, а также возможные отклонения генераторов тактовой частоты друг от друга, т. к. \textit{Time of Flight} очень чувствителен к погрешностям: маленькие отклонения времени дают большие отклонения расстояния;
    \item необходимо разработать быстрый и стабильный протокол общения передатчика и приёмника друг с другом.
\end{itemize}

В конечном итоге, необходимо выбрать оборудование с минимальными задержками, вычислить постоянную аппаратную задержку и описать программную модель измерения расстояния методом \textit{Time of Flight}.

Для устранение ошибки рассинхронизации <<часов>> приёмника и источника применяют метод \textit{TWTT (Two-way time transfer)}. Этот метод снижения ошибки измерения подробно описан в главе~\ref{sec:mitigation} <<\nameref{sec:mitigation}>>.

В работе рассматривается возможность построения <<Time of Flight>> устройства на микроконтроллерной базе, а также все препятствия на пути реализации такого устройства и погрешности измерений.
