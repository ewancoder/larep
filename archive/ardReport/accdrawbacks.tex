При передаче радиоволны с одного устройства на другое, замеренное время будет кратно тактовой частоте устройства, что ограничивает применение метода <<накопления>>. Время выполнения одного такта \textbf{работы} устройства должно быть равно или меньше времени полёта волны от одного устройства к другому. Например, если время полёта волны будет 3.3 нс, а такт процессора устройства будет занимать 33.3 нс, мы получим погрешность (задержку) в 10 раз больше чем само время Time of Flight. И т. к. время TOF будет <<ожидать>> следующий такт процессора чтобы <<быть отмеченным>>, мы получим данные с погрешностью горзадо более большой (в 10 раз больше) чем мы могли бы ожидать. Для частоты работы Arduino (16 МГц) мы получим дискретизацию в 20 метров независимо от того, какие приёмопередатчики мы будем использовать и на каком расстоянии будем измерять TOF.
