Радиолокацией называется обнаружение, определение координат и параметров движения различных объектов (целей), отражающих, переизлучающих или излучающих электромагнитную энергию (радиоволны). Термин <<локация>> происходит от латинского location – размещение, расположение. Комплекс радиотехнических устройств, выполняющих указанную задачу, представляет собой радиолокационную станцию (РЛС), или радиолокатор.

Радиолокация основана на следующих физических явлениях~\cite{wiki:radiolocation}:

\begin{enumerate}
    \item Радиоволны рассеиваются на встретившихся на пути их распространения электрических неоднородностях (объектами с другими электрическими свойствами, отличными от свойств среды распространения). При этом отражённая волна, также, как и собственно, излучение цели, позволяет обнаружить цель.
    \item На больших расстояниях от источника излучения можно считать, что радиоволны распространяются прямолинейно и с постоянной скоростью, благодаря чему имеется возможность измерять дальность и угловые координаты цели (Отклонения от этих правил, справедливых только в первом приближении, изучает специальная отрасль радиотехники --- Распространение радиоволн. В радиолокации эти отклонения приводят к ошибкам измерения).
    \item Частота принятого сигнала отличается от частоты излучаемых колебаний при взаимном перемещении точек приёма и излучения (эффект Доплера), что позволяет измерять радиальные скорости движения цели относительно РЛС.
    \item Пассивная радиолокация использует излучение электромагнитных волн наблюдаемыми объектами, это может быть тепловое излучение, свойственное всем объектам, активное излучение, создаваемое техническими средствами объекта, или побочное излучение, создаваемое любыми объектами с работающими электрическими устройствами.
\end{enumerate}

Основа радиолокации --- нахождение прямолинейного расстояния между объектами. Например, если найти расстояние от одного и того же объекта к двум различным <<искателям>>, можно будет аналитически рассчитать угол и относительно точное местоположение объекта.
