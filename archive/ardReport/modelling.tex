В процессе моделирования были проведены следующие операции:

\begin{itemize}
    \item передача радиосообщения с одного микроконтроллерного модуля на другой, и обратно;
    \item расчёт времени Time of Flight без учёта задержек оборудования: из-за недостаточной частоты оборудования, было протестировано определение времени отправки и принятия радиосообщения; сама радиоволна перемещается в пространстве на порядок быстрее работы используемого оборудования;
    \item общение микроконтроллерного модуля и ЭВМ с использованием модульной библиотеки (DLL), разработанной с возможностью встраивания в диспетчерские системы.
\end{itemize}

Общие задержки работы оборудования, полученные экспериментально, представлены в главе~\ref{sec:taskfreq}.
