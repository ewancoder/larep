Как было показано в главах~\ref{sec:taskfreq} и~\ref{sec:error}, чем ниже частота работы конечного устройства, тем больше будет квадратичная ошибка измерения. Для целей Time of Flight измерения будут больше подходить микропроцессорные устройства с большей тактовой частотой, использующие датчики на частотах 2.4 ГГц (Wi-Fi). Частота 2.4 ГГц является открытой частотой, также как и 433 МГц. Однако возникает проблема интерференции с другими устройствами, т. к. большинство устройств в современном мире работают на частотах 2.4 ГГц и будут вмешиваться в сигнал разрабатываемого Time of Flight устройства. Также стоимость микропроцессорных высокочастотных устройств зачастую превышает стоимость микроконтроллеров.

В разработанной модели использовались микроконтроллеры в связи с наличием и простоте использования, однако при разработке конечного устройства будет целесообразнее использовать микропроцессорное устройство.
