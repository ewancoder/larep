Для средств моделирования устройства выбирается библиотека VirtualWire, однако стоит рассмотреть вариант создания собственной, более быстрой библиотеки для общения между двумя микропроцессорными модулями.

На радиопринимающей стороне необходимо отфильтровать актуальные данные от постоянно находящихся на ней шумов, т. е. необходимо <<сконструировать>> программный детектор, способный принять пакет данных. Для этого необходимо разработать специфичный протокол передачи данных, где сообщению будет предшествовать некая <<шапка>>, состоящая из определённого количества бит. Эта шапка будет детектироваться приёмником, и всё что идёт после неё будет считаться сообщением (определённой длины). Для увеличения надёжности посылки, имеет смысл продублировать сообщение в обратном порядке, или же воспользоваться другими существующими алгоритмами повышения надёжности передачи (например, Cyclic Redundancy Check CRC-16).
