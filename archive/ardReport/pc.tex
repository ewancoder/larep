Модуль управления представляет собой динамическую библиотеку (dll) с набором методов и внешним API, способную встраиваться в любую SCADA- или другую систему. В процессе создания решения был написан стандартный графический интерфейс для удобного моделирования разных частей проекта, использующий dll-библиотеку.

Встраиваемая библиотека имеет следующие функции:

\begin{itemize}
    \item включение и отключение режима <<сервер>>;
    \item вывод расчитанного расстояния между сервером и клиентом (на экран).
\end{itemize}

Исходный код DLL-модуля представлен ниже:

\begin{minted}[gobble=4,fontsize=\footnotesize]{csharp}
    using System.IO.Ports;

    namespace TOF
    {
        public class TOF
        {
            public static string GetFirstPort()
            {
                try
                {
                    return SerialPort.GetPortNames()[0];
                }
                catch { return ""; }
            }

            private SerialPort sp;
            private bool server = false;
            public bool ServerMode
            {
                get { return server; }
                set
                {
                    try
                    {
                        server = ServerMode;
                        if (server)
                            sp.Write("s");
                        else
                            sp.Write("c");
                    }
                    catch { }
                }
            }
            public TOF(string portName = "COM1")
            {
                try
                {
                    sp = new SerialPort(portName, 115200);
                    sp.Open();
                }
                catch { }
            }

            public string Read()
            {
                try
                {
                    if (sp.BytesToRead != 0)
                        return sp.ReadExisting();
                    else
                        return "";
                }
                catch { return ""; }
            }
        }
    }
\end{minted}

Программа управления процессом моделирования имеет следующие возможности:

\begin{itemize}
    \item отображение анимации передачи и приёма пакета;
    \item отображение информации последовательного COM-порта;
    \item управляющие функции (функции встраиваемой библиотеки).
\end{itemize}

Исходный код программы управления представлен ниже:

\input{appv}
