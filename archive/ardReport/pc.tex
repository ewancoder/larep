Модуль управления представляет собой динамическую библиотеку (dll) с набором методов и внешним API, способную встраиваться в любую SCADA- или другую систему. В процессе создания решения был написан стандартный графический интерфейс для удобного моделирования разных частей проекта, использующий dll-библиотеку.

Встраиваемая библиотека имеет следующие функции:

\begin{itemize}
    \item включение и отключение режима <<сервер>>;
    \item вывод расчитанного расстояния между сервером и клиентом (на экран).
\end{itemize}

Исходный код DLL-модуля представлен ниже:

\begin{minted}[gobble=4,fontsize=\footnotesize]{csharp}
    using System.IO.Ports;

    namespace TOF
    {
        public class TOF
        {
            public static string GetFirstPort()
            {
                try
                {
                    return SerialPort.GetPortNames()[0];
                }
                catch { return ""; }
            }

            private SerialPort sp;
            private bool server = false;
            public bool ServerMode
            {
                get { return server; }
                set
                {
                    try
                    {
                        server = ServerMode;
                        if (server)
                            sp.Write("s");
                        else
                            sp.Write("c");
                    }
                    catch { }
                }
            }
            public TOF(string portName = "COM1")
            {
                try
                {
                    sp = new SerialPort(portName, 115200);
                    sp.Open();
                }
                catch { }
            }

            public string Read()
            {
                try
                {
                    if (sp.BytesToRead != 0)
                        return sp.ReadExisting();
                    else
                        return "";
                }
                catch { return ""; }
            }
        }
    }
\end{minted}

Программа управления процессом моделирования имеет следующие возможности:

\begin{itemize}
    \item отображение анимации передачи и приёма пакета;
    \item отображение информации последовательного COM-порта;
    \item управляющие функции (функции встраиваемой библиотеки).
\end{itemize}

Исходный код программы управления представлен ниже:

\begin{minted}[gobble=4,fontsize=\footnotesize,breaklines=true]{csharp}
    // Program.cs

    using System;
    using System.Windows.Forms;

    namespace TimeOfFlight
    {
        static class Program
        {
            [STAThread]
            static void Main()
            {
                Application.EnableVisualStyles();
                Application.Run(new MainForm());
            }
        }
    }

    // MainForm.cs
    using System;
    using System.Windows.Forms;

    namespace TimeOfFlight
    {
        public partial class MainForm : Form
        {
            TOF.TOF tof;

            public MainForm()
            {
                InitializeComponent();
                inputPort.Text = TOF.TOF.GetFirstPort();
                tof = new TOF.TOF("");
            }

            private void btnConnect_Click(object sender, EventArgs e)
            {
                tof = new TOF.TOF(inputPort.Text);
            }

            private void timerMonitor_Tick(object sender, EventArgs e)
            {
                boxMonitor.AppendText(tof.Read());
            }
        }
    }

    // MainForm.Designer.cs
    namespace TimeOfFlight
    {
        partial class MainForm
        {
            /// <summary>
            /// Required designer variable.
            /// </summary>
            private System.ComponentModel.IContainer components = null;

            /// <summary>
            /// Clean up any resources being used.
            /// </summary>
            protected override void Dispose(bool disposing)
            {
                if (disposing && (components != null))
                {
                    components.Dispose();
                }
                base.Dispose(disposing);
            }

            #region Windows Form Designer generated code

            /// <summary>
            /// Required method for Designer support - do not modify
            /// the contents of this method with the code editor.
            /// </summary>
            private void InitializeComponent()
            {
                this.components = new System.ComponentModel.Container();
                this.inputPort = new System.Windows.Forms.TextBox();
                this.btnConnect = new System.Windows.Forms.Button();
                this.boxMonitor = new System.Windows.Forms.RichTextBox();
                this.timerMonitor = new System.Windows.Forms.Timer(this.components);
                this.SuspendLayout();
                // 
                // inputPort
                // 
                this.inputPort.Dock = System.Windows.Forms.DockStyle.Top;
                this.inputPort.Location = new System.Drawing.Point(0, 0);
                this.inputPort.Name = "inputPort";
                this.inputPort.Size = new System.Drawing.Size(784, 20);
                this.inputPort.TabIndex = 0;
                this.inputPort.Text = "COM1";
                // 
                // btnConnect
                // 
                this.btnConnect.Dock = System.Windows.Forms.DockStyle.Top;
                this.btnConnect.Location = new System.Drawing.Point(0, 20);
                this.btnConnect.Name = "btnConnect";
                this.btnConnect.Size = new System.Drawing.Size(784, 23);
                this.btnConnect.TabIndex = 1;
                this.btnConnect.Text = "Connect";
                this.btnConnect.UseVisualStyleBackColor = true;
                this.btnConnect.Click += new System.EventHandler(this.btnConnect_Click);
                // 
                // boxMonitor
                // 
                this.boxMonitor.Dock = System.Windows.Forms.DockStyle.Fill;
                this.boxMonitor.Location = new System.Drawing.Point(0, 43);
                this.boxMonitor.Name = "boxMonitor";
                this.boxMonitor.ReadOnly = true;
                this.boxMonitor.Size = new System.Drawing.Size(784, 519);
                this.boxMonitor.TabIndex = 2;
                this.boxMonitor.Text = "";
                // 
                // timerMonitor
                // 
                this.timerMonitor.Enabled = true;
                this.timerMonitor.Tick += new System.EventHandler(this.timerMonitor_Tick);
                // 
                // MainForm
                // 
                this.AutoScaleDimensions = new System.Drawing.SizeF(6F, 13F);
                this.AutoScaleMode = System.Windows.Forms.AutoScaleMode.Font;
                this.ClientSize = new System.Drawing.Size(784, 562);
                this.Controls.Add(this.boxMonitor);
                this.Controls.Add(this.btnConnect);
                this.Controls.Add(this.inputPort);
                this.Name = "MainForm";
                this.Text = "Time of Flight model";
                this.WindowState = System.Windows.Forms.FormWindowState.Maximized;
                this.ResumeLayout(false);
                this.PerformLayout();

            }

            #endregion

            private System.Windows.Forms.TextBox inputPort;
            private System.Windows.Forms.Button btnConnect;
            private System.Windows.Forms.RichTextBox boxMonitor;
            private System.Windows.Forms.Timer timerMonitor;
        }
    }
\end{minted}

