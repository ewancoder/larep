Библиотека VirtualWire представляет собой высокоуровневую надстройку над Arduino для удобной передачи сообщений на расстоянии путём радиосигнала. Так как сама библиотека сложна и содержит множество задержек, замерим непосредственно минимальное время отправки сигнала --- максимальную частоту работы VirtualWire.

В целях эксперимента напишем программу, которая будет посылать один ASCII-символ данных типа \textbf{char}, занимающий ровно 1 байт (8 бит). Для обеспечения максимальной скорости, не будем ждать пока сообщение будет полностью отправлено, а будем сразу посылать следующее (если это позволит нам сделать сама \textit{VirtualWire}):

\begin{minted}[gobble=4,fontsize=\footnotesize]{c}
    #include <VirtualWire.h>
    unsigned long initial = 0; //0 - 4,294,967,295
    unsigned long final = 0;
    int i; //-32,768 - 32,767

    int TX_PIN = 1;

    const char *msg = "1";
    uint8_t buf[VW_MAX_MESSAGE_LEN];
    uint8_t buflen = VW_MAX_MESSAGE_LEN;

    void setup(){
        Serial.begin(9600);

        vw_set_tx_pin(TX_PIN);
        vw_set_ppt_inverted(true);
        vw_setup(2000);
    }

    void loop(){
        initial = micros();
        for (i = 0; i < 10; i++){
            vw_send((uint8_t *)msg, strlen(msg));
            //vw_wait_tx(); //Wait for message gone
        }
        final = micros();
        Serial.println(final - initial);
        while (final > 1E7); //Stop after 10 seconds
    }
\end{minted}

На выходе получается:

\begin{longtable}[c]{|c|c|}
    \caption{Результат использования VirtualWire}
    \label{VirtualWireResult}\\
    \hline
    \textbf{Количество, шт} & \textbf{Результат, мкс}\\
    \hline
    \endfirsthead
    \hline
    \textbf{Количество, шт} & \textbf{Результат, мкс}\\
    \hline
    \endhead
        7 & 480536\\
        \hline
        6 & 480540\\
        \hline
        7 & 480544\\
        \hline
\end{longtable}

Можно заметить, что в этом испытании было проведено лишь 10 итераций, что в итоге даёт нам частоту, не превышающую:

\begin{equation}
    \label{eq:freq3}
    f = \frac{n}{t} = \frac{10}{480540 \cdot 10^{-6}} = 20~\textrm{Гц}
\end{equation}

По-сути, VirtualWire способна передавать данные с максимальными скоростями вплоть до скоростей датчиков (9600 Кбит/с для FS1000A), однако это скорость передачи цельного блока данных. При передачи одного байта задержки складываются из-за больших битовых затрат на инициализацию драйвера передачи данных, формирование <<шапки>> пакета и проверки надежности доставки данных.

Таким образом, для наших целей библиотека VirtualWire не подходит.
