Есть несколько способов определения местоположения объекта:

\begin{itemize}
    \item ультразвуковые (sonic) устройства: просты в исполнении, однако на практике требуют прямой видимости между измерителем и объектом, что сокращает область применения;
    \item радиочастотные, методом определения уровня сигнала: не требуют больших ресурсов, однако точность такого измерения невелика: такой метод (RSSI) интерферирует с другими радиосигналами и ослабляется, отражаясь от препятствий;
    \item радиочастотные, прямой связью (метод \textit{Time of Flight}) между устройствами: измеряя время пролёта электромагнитной волны от передатчика к приёмнику и обратно, можно получить относительно точное измерение в рамках рабочей частоты устройства.
\end{itemize}

\subsubsection{Индикация уровня принимаемого сигнала}

Ультразвуковые и другие решения в большинстве случаев не подходят ввиду сложности помещений, цехов, складов. Наиболее простым способом измерения расстояния является метод \textit{Received Strength Signal Indication (RSSI)}. Любой беспроводной канал по стандарту IEEE 802.15.4 имеет протокольную функцию оценки качества связи (Link Quality Indicator), действие которой сводится к определению мощности принятого сигнала. Результат этого измерения можно вывести, откалибровать по известному расстоянию и оценить дальность до источника. Измерение расстояния производится следующим образом. Приемник с логарифмической амплитудной характеристикой принимает сигналы, по которым встроенный индикатор RSSI формирует 8-разрядный код RSSIVAL. Этот код получается в результате усреднения по восьми периодам (128 мкс) принятого сигнала и снабжается битом состояния, указывающим, когда RSSIVAL является валидным (т. е. приёмник имел возможность принять по крайней мере восемь периодов). Мощность принятого сигнала Р (дБм) вычисляется по формуле~\eqref{eq:rssi}:

\begin{equation}
    \label{eq:rssi}
    P = RSSI_{VAL} + RSSI_{OFFSET},
\end{equation}

где $RSSI_{OFFSET}$ — эмпирически подбираемая постоянная (порядка -45 дБм).

В идеальных условиях мощность обратно пропорциональна квадрату расстояния, логарифм мощности пропорционален расстоянию с некоторым коэффициентом, который устанавливается также эмпирически. Данный подход реализован в микроконтроллерах ZigBee фирмы TI серии CC2431.

Однако этому методу присущ ряд существенных ограничений, поскольку уровень сигнала является весьма изменчивым параметром из-за влияния следующих факторов:

\begin{itemize}
    \item быстрые и медленные замирания сигналов на трассе из-за изменения условий распространения радиоволн;
    \item многолучевое распространение вследствие отражений от различных металлических предметов;
    \item разброс выходной мощности передатчиков и чувствительности приёмников;
    \item влияние ориентации антенн из-за неравномерности диаграммы направленности.
\end{itemize}

Из-за воздействия указанных факторов реальная зависимость мощности от расстояния оказывается нелинейной и непостоянной во времени, вследствие чего точность измерений быстро падает с ростом расстояния. 

\subsubsection{Time of Flight}

Другой подход основан на измерении времени прохождения (пролета) сигнала (Time of Flight). Роутер посылает запрос на другой узел, получает ответный сигнал и определяет время его задержки. Полная задержка складывается из аппаратных задержек при обработке принятого и формировании ответного сигналов и времени распространения между узлами. Поскольку технические задержки известны с хорошей точностью, то их можно вычесть из полного значения, и оставшаяся величина будет характеризовать время пролета сигнала туда и обратно. Умножив половину времени задержки на скорость света, получим расстояние между узлами сети. В этом методе обеспечивается линейная связь между расстоянием и измеряемой величиной, и абсолютная точность измерения не зависит от расстояния. Для повышения точности используют многократные повторения процедуры измерения. Реально этот метод эффективен в полном диапазоне дальности работы сети (обычно сотни метров).

При качественной реализации, метод Time of Flight выигрывает у своих соперников почти по всем параметрам. Радиометки достаточно невелики и могут размещаться на объектах не занимая большого объема в отличие от ультразвуковых систем, которые занимают определённое место своим оборудованием.
