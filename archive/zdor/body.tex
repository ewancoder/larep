\section*{Введение}
\addcontentsline{toc}{section}{Введение}

Сознательная целенаправленная деятельность по формированию и развитию знания регулируется нормами и правилами, руководствуется определенными методами и приемами. Выявление и разработка таких норм, правил, методов и приемов, которые представляют собой не что иное, как аппарат сознательного контроля, регулирования деятельности по формированию и развитию научного знания, составляют предмет логики и методологии научного познания. При этом термин ``логика'' традиционно связывается с выявлением и формулировкой правил вывода одних знаний из других, правил определения понятий, что, начиная еще с античности, составляло предмет формальной логики. В настоящее время разработка логических норм рассуждения, доказательства и определения как правил работы с предложениями и терминами языка науки осуществляется на основе аппарата современной математической логики \cite{mir:philosophy}.

Конкретные методы, как эмпирические, так и теоретические, сопровождаются проведением логических процедур. Эффективность эмпирических и теоретических методов находится в прямой зависимости от того, насколько правильно с точки зрения логики строятся соответствующие научные рассуждения \cite{lekt:logic}.

\section{Теоретическая основа}

Теоретическую основу всех форм методологического исследования научного познания в целом составляет философско-гносеологический уровень анализа науки. Его специфика заключается в том, что научное познание рассматривается здесь в качестве элемента более широкой системы --- познавательной деятельности в ее отношении к объективному миру, в ее включенности в практически-преобразовательную деятельность человека. Теория познания --- не просто общая наука о познании, это философское учение о природе познания \cite{mir:philosophy}.

Гносеология выступает как теоретическое основание различных специально-научных форм методологического анализа, тех его уровней, где исследование научного познания осуществляется уже нефилософскими средствами. Она показывает, что, только понимая познание как формирование и развитие идеального плана человеческой практически-преобразующей деятельности, можно анализировать коренные свойства познавательного процесса, сущность знания вообще и его различных форм, в том числе и научного знания. Вместе с тем в настоящее время не только само научное познание, но и его философско-гносеологическую проблематику невозможно анализировать, не привлекая материала из более специальных разделов методологии науки. Скажем, философский анализ проблемы истины в науке предполагает рассмотрение средств и методов эмпирического обоснования научного знания, специфических особенностей и форм активности субъекта научного познания, роли и статуса теоретических идеализированных конструкций и пр \cite{mir:philosophy}.

Любая форма исследования научного знания (даже если она ориентирована непосредственно на внутренние проблемы специальной науки) потенциально содержит в себе зародыши философской проблематики. Она неявно опирается на предпосылки, которые при их осознании и превращении в предмет анализа в конечном счете предполагают определенные философские позиции \cite{mir:philosophy} \cite{mir:philosophy}.

\section{Методы научного познания}

Одна из основных задач методологического анализа заключается в выявлении и изучении методов познавательной деятельности, осуществляемой в науке, в определении возможностей и пределов применимости каждого из них. В своей познавательной деятельности, в том числе и в научной, люди осознанно или неосознанно используют самые разнообразные методы. Ясно, что осознанное применение методов, основанное на понимании их возможностей и границ, придает деятельности человека большую рациональность и эффективность.

Методологический анализ процесса научного познания позволяет выделить два типа приемов и методов исследования. Во-первых, приемы и методы, присущие человеческому познанию в целом, на базе которых строится как научное, так и обыденное знание. К ним можно отнести анализ и синтез, индукцию и дедукцию, абстрагирование и обобщение и т. д. Назовем их условно общелогическими методами. Во-вторых, существуют особые приемы, характерные только для научного познания, --- научные методы исследования. Последние, в свою очередь, можно подразделить на две основные группы: методы построения эмпирического знания и методы построения теоретического знания.

С помощью общелогических методов познание постепенно, шаг за шагом, раскрывает внутренние существенные признаки предмета, связи его элементов и их взаимодействие друг с другом. Для того чтобы осуществить эти шаги, необходимо целостный предмет расчленить (мысленно или практически) на составляющие части, а затем изучить их, выделяя свойства и признаки, прослеживая связи и отношения, а также выявляя их роль в системе целого. После того как эта познавательная задача решена, части вновь можно объединить в единый предмет и составить себе конкретно-общее представление, т. е. такое представление, которое опирается на глубокое знание внутренней природы предмета. Эта цель достигается с помощью таких операций, как анализ и синтез \cite{mir:philosophy}.

Научными методами называется совокупность практических и мыслительных действий, обеспечивающих получение, систематизацию и обоснование научных знаний. Методы – «технология выработки» научных знаний. В любой сфере деятельности технологическое знание является необходимым условием успеха. Знание о методах, приемах научной деятельности называется методологией. Методологическое знание существует в различных формах – как неявное знание, присутствующее в реальных процессах научного познания; как методологическая рефлексия ученого, применяющего тот или иной метод, или оценивающего его результаты; как раздел философии науки, направленный на познание деятельности ученого и т.д.. Методологическая рефлексия – необходимый элемент обоснования научных знаний: их надежность зависит от надежности применяемых методов познания \cite{stud:methods}.

\subsection{Общелогические методы}

\textbf{Анализ} --- это расчленение целостного предмета на составляющие части (стороны, признаки, свойства или отношения) с целью их всестороннего изучения.

\textbf{Синтез} --- это соединение ранее выделенных частей (сторон, признаков, свойств или отношений) предмета в единое целое.

Анализ и синтез --- наиболее элементарные и простые приемы познания, которые лежат в самом фундаменте человеческого мышления. Вместе с тем они являются и наиболее универсальными приемами, характерными для всех его уровней и форм \cite{mir:philosophy}.

\textbf{Абстрагирование} --- это прием мышления, который заключается в отвлечении от ряда свойств и отношений изучаемого явления с одновременным выделением интересующих нас свойств и отношений. Результатом абстрагирующей деятельности мышления является образование различного рода абстракций, которыми являются как отдельно взятые понятия и категории, так и их системы.

\textbf{Обобщение} --- это операция, которая осуществляется как переход от частного или менее общего понятия и суждения к более общему понятию или суждению. Например, такие понятия, как ``клен'', ``липа'', ``береза'' и т. д., являются первичными обобщениями, от которых можно перейти к более общему понятию ``лиственное дерево''.

\textbf{Индукция} --- это такой метод исследования и способ рассуждения, в котором общий вывод строится на основе частных посылок. Основой индукции являются опыт, эксперимент и наблюдение, в ходе которых собираются отдельные факты. Затем, изучая эти факты, анализируя их, мы устанавливаем общие и повторяющиеся черты ряда явлений, входящих в определенный класс. На этой основе строится индуктивное умозаключение, в качестве посылок которого выступают суждения о единичных объектах и явлениях с указанием их повторяющегося признака и суждение о классе, включающем данные объекты и явления. В качестве вывода получают суждение, в котором признак приписывается всему классу.

\textbf{Дедукция} --- это способ рассуждения, посредством которого из общих посылок с необходимостью следует заключение частного характера. Дедукция отличается от индукции прямо противоположным ходом движения мысли. В дедукции, как это видно из определения, опираясь на общее знание, делают вывод частного характера. Одной из посылок дедукции обязательно является общее суждение. Если оно получено в результате индуктивного рассуждения, тогда дедукция дополняет индукцию, расширяя объем нашего знания.

\textbf{Аналогия} --- это такой прием познания, при котором на основе сходства объектов в одних признаках заключают об их сходстве и в других признаках. Так, при изучении природы света были установлены такие явления, как дифракция и интерференция. Эти же свойства ранее были обнаружены у звука и вытекали из его волновой природы. На основе этого сходства X. Гюйгенс заключил, что и свет имеет волновую природу.

Умозаключения по аналогии, понимаемые предельно широко, как перенос информации об одних объектах на другие, составляют гносеологическую основу \textbf{моделирования}.

\textbf{Моделирование} --- это изучение объекта (оригинала) путем создания и исследования его копии (модели), замещающей оригинал с определенных сторон, интересующих познание. Модель всегда соответствует объекту --- оригиналу --- в тех свойствах, которые подлежат изучению, но в то же время отличается от него по ряду других признаков, что делает модель удобной для исследования интересующего нас объекта.

На современном этапе научно-технического прогресса большое распространение в науке и в различных областях практики получило компьютерное моделирование. Компьютер, работающий по специальной программе, способен моделировать самые различные реальные процессы (например, колебания рыночных цен, рост народонаселения, взлет и выход на орбиту искусственного спутника Земли, химическую реакцию и т. д.) \cite{mir:philosophy}.

\subsection{Эмпирические методы}

\textbf{Наблюдение} --- это целенаправленное восприятие явлений объективной действительности, в ходе которого мы получаем знание о внешних сторонах, свойствах и отношениях изучаемых объектов. Процесс научного наблюдения является не пассивным созерцанием мира, а особого вида деятельностью, которая включает в себя в качестве элементов самого наблюдателя, объект наблюдения и средства наблюдения. Важнейшей особенностью наблюдения является его целенаправленный характер. Эта целенаправленность обусловлена наличием предварительных идей, гипотез, которые ставят задачи наблюдению \cite{mir:philosophy}.

Наблюдение как метод эмпирического исследования всегда связано с описанием, которое закрепляет и передает результаты наблюдения с помощью определенных знаковых средств. \textbf{Эмпирическое описание} --- это фиксация средствами естественного или искусственного языка сведений об объектах, данных в наблюдении. С помощью описания чувственная информация переводится на язык понятий, знаков, схем, рисунков, графиков и цифр, принимая тем самым форму, удобную для дальнейшей рациональной обработки (систематизации, классификации и обобщения).

\textbf{Описание} подразделяется на два основных вида --- \textbf{качественное} и \textbf{количественное}.

\textbf{Количественное} описание осуществляется с применением языка математики и предполагает проведение различных измерительных процедур. В узком смысле слова его можно рассматривать как фиксацию данных измерения. В широком смысле оно включает также нахождение эмпирических зависимостей между результатами измерений. Лишь с введением метода измерения естествознание превращается в точную науку. В основе операции измерения лежит сравнение объектов по каким-либо сходным свойствам или сторонам. Чтобы осуществить такое сравнение, необходимо иметь определенные единицы измерения, наличие которых дает возможность выразить изучаемые свойства со стороны их количественных характеристик. В свою очередь, это позволяет широко использовать в науке математические средства и создает предпосылки для математического выражения эмпирических зависимостей. Сравнение используется не только в связи с измерением. В ряде подразделений науки (например, в биологии, языкознании) широко используются сравнительные методы.

Активное вмешательство исследователя в протекание природного процесса, искусственное создание им условий взаимодействия отнюдь не означает, что экспериментатор сам, по своему произволу творит свойства предметов, приписывает их природе. Ни радиоактивность, ни световое давление, ни условные рефлексы не являются свойствами, выдуманными или изобретенными исследователями, но они выявлены в экспериментальных ситуациях, созданных самим человеком. Его творческая способность проявляется лишь в создании новых комбинаций природных объектов, в результате которых выявляются скрытые, но объективные свойства самой природы.

\textbf{Формализация} --- это прием, заключающийся в построении абстрактно-математических моделей, раскрывающих сущность изучаемых процессов действительности. При формализации рассуждения об объектах переносятся в плоскость оперирования со знаками (формулами). Отношения знаков заменяют собой высказывания о свойствах в отношениях предметов. Таким путем создается обобщенная знаковая модель некоторой предметной области, позволяющая обнаружить структуру различных явлений и процессов при отвлечении от качественных характеристик последних.

\textbf{Аксиоматический метод} --- метод, где сначала задается набор исходных положений, не требующих доказательства (по крайней мере, в рамках данной системы знания). Эти положения называются \textbf{аксиомами} или \textbf{постулатами}. Затем из них по определенным правилам строится система выводных предложений. Совокупность исходных аксиом и выведенных на их основе предложений образует аксиоматически построенную теорию.

\textbf{Аксиомы} --- это утверждения, доказательства истинности которых не требуется. Логический вывод позволяет переносить истинность аксиом на выводимые из них следствия.

При формальном построении аксиоматической системы уже не ставится требование выбирать только интуитивно очевидные аксиомы, для которых заранее задана область характеризуемых ими объектов. Аксиомы вводятся формально, как описание некоторой системы отношений: термины, фигурирующие в аксиомах, первоначально определяются только через их отношение друг к другу. Тем самым аксиомы в формальной системе рассматриваются как своеобразные определения исходных понятий (терминов). Другого, независимого определения указанные понятия первоначально не имеют.

Дальнейшее развитие аксиоматического метода привело к третьей стадии --- построению \textbf{формализованных аксиоматических систем}. Формальное рассмотрение аксиом дополняется на этой стадии использованием математической логики как средства, обеспечивающего строгое выведение из них следствий. В результате аксиоматическая система начинает строиться как особый формализованный язык (исчисление). Вводятся исходные знаки --- термины, затем указываются правила их соединения в формулы, задается перечень исходных принимаемых без доказательства формул и, наконец, правила вывода из основных формул производных. Так создается \textbf{абстрактная знаковая модель}, которая затем интерпретируется на самых различных системах объектов.

В отличие от математики и логики в эмпирических науках теория должна быть не только непротиворечивой, но и обоснованной опытным путем. Отсюда возникают особенности построения теоретических знаний в эмпирических науках. Специфическим приемом такого построения и является \textbf{гипотетико-дедуктивный метод}, сущность которого заключается в создании системы дедуктивно связанных между собой гипотез, из которых в конечном счете выводятся утверждения об эмпирических фактах. Теория, создаваемая гипотетико-дедуктивным методом, может шаг за шагом пополняться гипотезами, но до определенных пределов, пока не возникают затруднения в ее дальнейшем развитии. В такие периоды становится необходимой перестройка самого ядра теоретической конструкции, выдвижение новой гипотетико-дедуктивной системы, которая смогла бы объяснить изучаемые факты без введения дополнительных гипотез и, кроме того, предсказать новые факты. Чаще всего в такие периоды выдвигается не одна, а сразу несколько конкурирующих гипотетико-дедуктивных систем.

Задача теоретического познания состоит в том, чтобы дать целостный образ исследуемого явления. Любое явление действительности можно представить как конкретное переплетение самых различных связей. Теоретическое исследование выделяет эти связи и отражает их с помощью определенных научных абстракций. Но простой набор таких абстракций не дает еще представления о природе явления, о процессах его функционирования и развития. Для того чтобы получить такое представление, необходимо мысленно воспроизвести объект во всей полноте и сложности его связей и отношений.

Такой прием исследования называется методом \textbf{восхождения от абстрактного к конкретному}. Применяя его, исследователь вначале находит главную связь (отношение) изучаемого объекта, а затем, шаг за шагом прослеживая, как она видоизменяется в различных условиях, открывает новые связи, устанавливает их взаимодействия и таким путем отображает во всей полноте сущность изучаемого объекта.

Все описанные методы познания в реальном научном исследовании всегда работают во взаимодействии. Их конкретная системная организация определяется особенностями изучаемого объекта, а также спецификой того или иного этапа исследования. В процессе развития науки развивается и система ее методов, формируются новые приемы и способы исследовательской деятельности. Задача методологии науки состоит не только в выявлении и фиксации уже сложившихся приемов и методов исследовательской деятельности, но и в выяснении тенденций их развития \cite{mir:philosophy}.

\subsection{Теоретические методы}

В целом, теоретическое знание обладает дедуктивной структурой, где можно выделить некоторые общие понятия, принципы и гипотезы, составляющие теоретический базис и систему вытекающих из этого базиса следствий. Отличительной особенностью развитых теорий является использование математического формализма, реализующегося в аксиоматизации и формализации теорий, построении математических моделей и математических гипотез. Использование математического аппарата является мощным средством современного научного познания. В то же время теоретическое знание имеет сложную структуру, и формально математическая часть представляет лишь одну из сторон теории, но не всю теорию. Кроме этой части, теория включает в себя особую идеализированную модель действительности, оперирование которой осуществляется в форме мысленного эксперимента. Элементами, из которых она состоит, являются так называемые абстрактные объекты, связи и отношения которых образуют данную модель. Наличие таких объектов, замещающих в познании реальные вещи, их свойства и отношения, является характерной особенностью теоретического знания. Теоретический язык описывает отношения абстрактных объектов теоретической модели, которая так или иначе связана с наблюдаемой реальностью. Благодаря этой связи теоретические высказывания обретают объективный смысл. В основании сложившейся теории всегда можно обнаружить взаимосогласованную сеть абстрактных объектов, определяющих специфику данной теории. Эту сеть можно представить как фундаментальную теоретическую схему — абстрактную идеализированную модель действительности, изучаемой в рамках теории. Вокруг неё формируются частные теоретические схемы, входящие в состав научной теории. Кроме указанной модели, внутри развитой теории можно выделить и другие подсистемы абстрактных объектов \cite{gtm:methods}.

Сложившаяся теория включает множество элементов, которые образуют структуру теории. Они фиксируются в особых языковых средствах: имеются высказывания, описывающие теоретическую схему, выражения, образующие математический аппарат; в состав теории входят также описания правил связи абстрактных объектов теоретической схемы с реальными объектами опыта и выражения, характеризующие указанные абстрактные объекты в терминах картины мира. Вся эта совокупность высказываний, связанных между собой, образует язык сложившейся научной теории.

Теория создается с целью объяснения какого-то класса явлений. Будучи построенной, она одновременно выступает и в функции объяснения, и в функции предсказания, которые тесно связаны друг с другом.

\textbf{Объяснение} является одной из наиболее важных задач научного знания. Именно в процессе объяснения раскрываются существенные стороны и отношения предметов, устанавливается внутренняя причинная взаимосвязь явлений и их закономерная обусловленность. Объяснить явление — значит установить его фундаментальные свойства и отношения, основную причинную обусловленность, выявить общие законы, которым оно подчиняется. С логической точки зрения объяснение представляет собой включение исследуемых объектов в систему теоретического знания, подведение их под общие положения и принципы науки, на основе чего достигается наиболее полное и глубокое понимание этих объектов \cite{gtm:methods}.

\section*{Заключение}
\addcontentsline{toc}{section}{Заключение}

Методы науки пластичны, изменчивы, требуют творческой изобретательности и такого воплощения, чтобы результат был ответом на определённый исследовательский вопрос. Ф. Бэкон называл научную опытную деятельность ``искусством задавать Природе вопросы''.

Помимо методов в научном познании находят реализацию более общие методологические процедуры. Это – общенаучные подходы. Если каждый метод требует соблюдения определенных, чётко сформулированных правил действия, то подходы имеют менее определенное содержание. Они не требуют выполнения жёстких предписаний, не регламентируют каждый шаг субъекта познания, а задают лишь общие направления исследования, его ориентацию на постижение той или иной характеристики бытия объекта. К общенаучным относится системный подход (направленность на постижение системного характера объекта), функциональный подход (ориентация на постижение функционирования объекта в том или ином контексте его существования), структурный подход, субстратный, деятельностный, информационный и др. подходы \cite{stud:methods}.

\bibliography{../web,../books}
