\pdfminorversion=4

\documentclass[10pt,pdf,hyperref={unicode}]{beamer} %class for presentations
\usepackage{cmap} %for PDF copyable font
\usepackage[T2A]{fontenc} %cyrillic symbols
\usepackage[utf8]{inputenc} %T2A encoding
\usepackage[english,russian]{babel} %russian chapters
\usepackage{pscyr} %quality fonts

\usetheme{Warsaw}
\usecolortheme{spruce}
\setbeamercolor{enumerate item}{bg=black,fg=yellow}
\setbeamertemplate{section in toc}{\color{black!70}\inserttocsectionnumber.~\inserttocsection}
\setbeamercolor{palette quaternary}{fg=MSUgreen!20}
\usefonttheme[onlylarge]{structurebold}
\usefonttheme[onlymath]{serif}
\setbeamerfont{institute}{size=\normalsize}
\setbeamercovered{transparent} %nonactive elements is hardly visible
\setbeamertemplate{navigation symbols}{}
\setbeamercolor{item projected}{bg=orange}

\setbeamertemplate{caption}{\raggedright\insertcaption\par}

%\usepackage{graphicx} %Pasting images
%\usepackage[justification=centering]{caption} %Redefine fig/tab caption
    %Centering all CAPTIONS by justification param
%    \DeclareCaptionLabelFormat{gostfigure}{Рисунок #2}
%    \DeclareCaptionLabelSeparator{gostsep}{~---~}
%    \captionsetup{labelsep=gostsep}
%    \captionsetup[figure]{labelformat=gostfigure}
%    \captionsetup{figurewithin=section}

%\usepackage{indentfirst} %first paragraph with indent

%\linespread{1.3} %1.5 interval
%\renewcommand{\rmdefault}{ftm} %Times New Roman
%\frenchspacing %Only one space between words

%\usepackage{geometry}
%    \geometry{left=2.5cm}
%    \geometry{right=1.5cm}
%    \geometry{top=2cm}
%    \geometry{bottom=2cm}

%\usepackage{fancyhdr} %colontitle package
%    \pagestyle{fancy} %set this style
%    \fancyhf{} %clear values
%    \fancyhead[R]{\thepage} %top numbering
%    \renewcommand{\headrulewidth}{0pt} %remove line

\usepackage{array} %for centering width-ed table cells
    \newcolumntype{C}[1]{>{\centering\let\newline\\\arraybackslash\hspace{0pt}}p{#1}}

\usepackage{longtable} %cool table

\usepackage{tikz}
\usepackage{tkz-euclide}
    \usetkzobj{all}
    \tikzset{
        right angle quadrant/.code={
            \pgfmathsetmacro\quadranta{{1,1,-1,-1}[#1-1]}     % Arrays for selecting quadrant
            \pgfmathsetmacro\quadrantb{{1,-1,-1,1}[#1-1]}},
        right angle quadrant=1, % Make sure it is set, even if not called explicitly
        right angle length/.code={\def\rightanglelength{#1}},   % Length of symbol
        right angle length=2ex, % Make sure it is set...
        right angle symbol/.style n args={3}{
            insert path={
                let \p0 = ($(#1)!(#3)!(#2)$) in     % Intersection
                    let \p1 = ($(\p0)!\quadranta*\rightanglelength!(#3)$), % Point on base line
                    \p2 = ($(\p0)!\quadrantb*\rightanglelength!(#2)$) in % Point on perpendicular line
                    let \p3 = ($(\p1)+(\p2)-(\p0)$) in  % Corner point of symbol
                (\p1) -- (\p3) -- (\p2)
            }
        }
    }
    
%\usepackage{titlesec} %For new page with each section
%\newcommand{\sectionbreak}{\clearpage}

\bibliographystyle{unsrt}
%\bibliographystyle{ugost2008} %Just like it goes in text, non-sorted

\makeatletter %make bibliography within dot-separator
\renewcommand{\@biblabel}[1]{#1.}
\makeatother

\usepackage{cite} %for [1-3], not [2,1,3]
\usepackage{url} %for url citations

\makeatletter
\setlength{\@fptop}{0pt}
\setlength{\@fpsep}{8pt}
\makeatother

\usepackage{multirow} %for multiROWs

\usepackage{colortbl} %for coloring table cells

%\usepackage{hhline} %for use \hhline instead of \cline, because "colortbl" package is not compatible with \cline

%\usepackage{subfig} %for \subfloat
%\renewcommand{\thesubfigure}{\asbuk{subfigure}}

%\usepackage{floatrow} %improvement for floats, centered by default, captions are below
%\floatsetup[table]{style=plaintop} %for table captions are above
    %I am using "longtable" so I don't need it
    %BUT AS I DON'T NEED FLOATROW YET, I'll use simple macro:

\makeatletter %changes the catcode of @ to 11
\g@addto@macro\@floatboxreset\centering
    %\g@addto@macro - LaTeX build-in macro which GLOBALLY ADDS #2 to #1
    %@floatboxreset - internal command for figure environment
\makeatother %changes the catcode of @ back to 12

\usepackage{amssymb} %for \land
\usepackage{amsmath} %for curly equation!

%\makeatletter %dotted toc
    %\renewcommand*\l@section{\@dottedtocline{1}{1.5em}{2.3em}}
%    \renewcommand*\l@section{\@dottedtocline{1}{0em}{1.3em}}
%\makeatother

%\usepackage[nottoc]{tocbibind} %for bibtex in toc

%\usepackage{pst-circ} %for electic circuits
\usepackage{circuitikz}
\usetikzlibrary{circuits.logic.IEC}

\input kvmacros %karnaugh macros

\usepackage{minted}

\makeatletter %I don't know what it does, but it definitely makes source code offsets look nicer :) (supposed to shorten offsets for itemize/enumerate items)
\renewcommand{\@listI}{\topsep=0pt}
\makeatother

\usepackage{subcaption} %for subcaptionbox support
\renewcommand{\thesubfigure}{\asbuk{subfigure}}
