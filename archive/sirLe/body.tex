\part{ЧПУ}

\section{Макропрограммы}

Функции, аналогичные подпрограммам, могут выполнять макропрограммы пользователя (макросы). Они позволяют использовать операции с переменными, арифметические и логические операции, могут вызываться также, как и подпрограммы --- простыми командами. В обычной программе обработки G-код и расстояния перемещения задаются непосредственно с помощью цифрового значения. В макропрограммах пользователя цифровые значения тоже могут задаваться непосредственно или с использованием номера переменной. Если используется номер переменной, то значение переменной может быть изменено программно или с помощью операций ручного ввода данных.

Переменные также обозначеются знаками \#, как и параметры. Для ввода номера переменной можно использовать выражение. В этом случае выражение должно быть заключено, например, в квадратные скобки. Переменные классифицируются на 4 типа по номеру переменной.

(это FANUC)

\begin{enumerate}
    \item \#0.
    \item \#1 \#33 --- локальные переменные; могут использоваться только внутри макропрограммы и содержат такие данные, например, как результаты операций. При отключении питания они не сохраняются (похоже на РОН).
    \item \#100 -- \#199, \#500 -- \#599 --- общие переменные. Общие переменные могут совместно использоваться в различных макропрограммах. При отключении питания переменные от 100 до 199 --- обнуляются, переменные от \#500 до \#599 --- сохраняются (похоже на память).
    \item От \#1000, системные переменные. Системные переменные используются для считывания и записи различных данных ЧПУ, например текущего положения и значений компенсации погрешности инструмента.
\end{enumerate}
    
Локальные и общие переменные могут иметь диапазон от $10^{-29}$ до $10^{-27}$. Системные переменные помимо для использования записи и чтения внутренних данных могут использоваться для автоматизации и разработки программ общего назначения. Некоторые системные переменные могут только считываться. С помощью системных переменных можно осуществлять обмен между УУ старком (ПЛК) и макропрограммами пользователя, т.е. реализовать интерфейс. В частности, переменные с номерами от \#1000 до \#1015 позволяют осуществить поразрядное считывание сигнала от ПЛК в макропрограмму пользователя. Для обратной передачи --- \#1100 -- \#1115.

Системные переменные будут также использоваться, например, для хранения коррекции на инструмент, запоминания сигналов тревоги и т.д.

С переменными можно выполнять операции арифметические и логические. Т.е. выполняемые операции арифметические: сумма, разность, произведение; тригонометрические, логические (или, не, и) и преобразования двоичного в двоично-десятичный и обратно.

В макропрограммах существует такое понятие как операторы. Операторы~--- это блоки, содержащие арифметические или логические операции, блоки, содержащие управляющий оператор (например goto), блоки, содержащие команду вызова макропрограммы. В программе можно изменить процесс управления с помощью оператора перехода goto и условного перехода if. Используются 3 типа переходов:

\begin{enumerate}
    \item Безусловный переход goto.
    \item Условный переход if.
    \item Оператор цикла while.
\end{enumerate}

Условный переход после оператора if указывается, например:

\begin{minted}[gobble=4,fontsize=\small]{tex}
    IF [#1 EQ #Q] GOTO 2
        ...

    N2 G00 G91 X.. Y..

\end{minted}

Если условие выполняется, то выполняется кадр 2, если не выполняется --- выполняется блок.

\begin{minted}[gobble=4,fontsize=\small]{tex}
    IF [#1 EQ #Q] THEN #3 = 0
\end{minted}

Вместо переменной может быть использовано выражение.

Используются операторы:

\begin{itemize}
    \item EQ --- равно;
    \item NE --- не равно;
    \item GT --- больше;
    \item GE --- больше либо равно;
    \item LT --- меньше;
    \item LE --- меньше либо равно.
\end{itemize}

Цикл \textbf{while}: если условие удовлетворяется, то выполняется программа от метки \textbf{do} до \textbf{end}. Если заданное условие не удовлетворяется, то выполнение переходит к блоку после \textbf{end}.

Могут быть организованы бесконечные циклы. Например, если задан оператор \textbf{do} и не задан оператор \textbf{while}.

\subsection{Вызов макропрограммы}

Для вызова макропрограммы можно использовать инструкцию G65. Вызов макропрограммы с помощью инструкции G65 отличается от вызова подпрограммы с помощью инструкции M98 следующими особенностями:

\begin{itemize}
    \item с помощью инструкции G65 можно задать аргумент, при этом данные передаются в макропрограмму. N98 не позволяет это сделать;
    \item если в блоке подпрограммы, вызванной M98, содержится другая программа, которая входит \textit{вложенный} вызов, тогда содержится условие и вызов подпрограммы выполняется по условию. G65 вызывает макропрограмму без условий;
    \item в некоторых случаях использование M98 может привести к остановке станка. При использовании G65 остановки станка быть не может;
    \item при выполнении инструкции G65 меняются значения локальных переменных. При M98 локальные переменные не меняются;
\end{itemize}

Для вызова макропрограммы с помощью G65 используется следующий формат:

\begin{equation}
    G65 PpLl,
\end{equation}

где

\begin{itemize}
    \item P --- номер вызываемой программы;
    \item l --- количество повторов (по умолчанию, 1);
    \item p --- аргумент.
\end{itemize}

В некоторых случаях используется модальный вызов макропрограммы (G66). В этом случае сначала выполняется блок основной программы, в котором задано перемещение, а затем осуществляется вызов макропрограммы. G67 --- отмена модального вызова.

Для чистовой обработки ЧПУ предварительно считывают оператор ЧПУ, подлежащий выполнению следующей. Эта операция называется ``буферизация''. В режиме управления с предварительным расширенным просмотром считывают несколько блоков. Также, в режиме коррекции на режущий инструмент (G41, G42) ЧПУ считывает предварительно операторы на 2 или 3 блока, чтобы найти точки пересечения даже если не находится в режиме управления с расширенным предварительным просмотром. Макрооператоры арифметических выражений и условные переходы обрабатываются с момента их считывания в буфер. В блоках, содержащих команды того или иного остановка (M00, M30) и в блоках содержащих M-коды, для которых буферизация прекращается специальными параметрами, буферизация останавливается, чтобы после этого произвести предварительную считку операторов ЧПУ.

\section{Оперативное программирование}

Оперативное программирование предполагает подготовку управляющей программы непосредственно с клавиатуры пульта УЧПУ. Такие УЧПУ (класс HNC) имеют, как правило, упрощённую схему ввода программы, большое количество постоянных циклов, ориентированных на станок, простую схему редактирования и ввода коррекции. Помимо ввода информации с пульта, может вводиться информация и с других носителей. Программирование обработки ведется отдельными кадрами в жёсткой последовательности в режиме редактирования и ввода программы. Каждый кадр обязательно предусматривает порядковый номер, типовой код, последовательность данных, определяемую типовым кодом. Типовой код определяет разновидность и способ интерпретации данных, записанных в кадре. В одном кадре может быть только один типовой код с индексом G. В памяти типовые коды располагают определенные последовательности, и в режиме программирования они последовательно извлекаются, причем процесс повторяется циклически. Ввод числовых данных производится с помощью специальных кнопок.

\section{Символьно-графическое программирование}

Метод символьно-графического программирования реализуется в режиме диалога с выводом графического изображения на дисплей и контроль процессов с помощью моделирования. Для его реализации необходимо прикладное математическое обеспечение и соответствующие технические свойства. Основа системы --- это библиотека подпрограмм, включающая данные по заготовкам, приспособлениям, инструментам, геометрическим элементам контура, режимам обработки и т.п. Специальный графический процессор позволяет формировать на экране различные графические изображения: обрабатываемый контур, заготовку, схематическое изображение приспособлений инструмента, схему удаления припуска и т.д. Все эти изображения могут быть совмещены отдельным элементом с заданной динамиков в режиме реального времени, что позволяет осуществлять символьно-графическое моделирование введенной программы и контроль текущего процесса обработки. Процесс программирования обработки при рассматриваемом методе ведется в режиме диалога по принципу ``меню'', когда оператору предлагается набор отдельных решений. Оператор в процессе программирования выбирает требуемый по чертежу вариант и вводит с пульта УЧПУ необходимые данные предлагаемой последовательности и по предлагаемой схеме. Программирование обработки включает последовательные этапы:

\begin{itemize}
    \item формирование геометрии заготовки и детали;
    \item конкретизацию технологических требований;
    \item определение стратегии и выбор схемы обработки и инструмента;
    \item определение режимов обработки;
    \item разработку схемы наладки;
    \item наладку;
    \item динамическое моделирование.
\end{itemize}

Геометрия заготовки и контура детали формируется на экране символьными клавишами, при этом же вводятся особые технологические требования. Эти требования вводят с клавиатуры УЧПУ специальными кодами.

Инструмент для обработки предлагается оператору автоматически по каждой выбранной схеме обработки. Оператор может модифицировать все движения, изменить, уточнить. Точно так же, режимы обработки могут быть предложены автоматически или назначены оператором. После определения всех параметров, система автоматически генерирует программу работы в формате ISO.

\part{SCADA-системы}

Диспетчер в многоуровневой автоматизированной системе управления технологическими процессами получает информацию с монитора ЭВМ или с электронной системы отображения информации и воздействует на объекты, находящиеся на значительном расстоянии от него с помощью телекоммуникационных систем, контроллеров, интеллектуальных исполнительных механизмов.

Необходимым условием эффективной реализации диспетчерского управления, имеющего ярко выраженный динамический характер, становится работа с информацией, т.е. процесс сбора, передачи, обработки, отображения, предоставления информации. В этих условиях от диспетчера требуется профессиональное знание технологического процесса, основ управления, опыт работы в информационных системах, умение принимать решение в диалоге с ЭВМ в нештатных и аварийных ситуациях.

Одна из причин значительно возрастающего технологического риска --- это неправильный подход к построению сложных систем управления, который заключается в применении современных технических и технологических достижений и недооценка эффективности человеко-машинного интерфейса. Для решения этой проблемы используется концепция \textbf{SCADA}.

Применение SCADA-технологии позволяет достичь высокого уровня автоматизации, решение задач разработки систем управления, сбора, обработки, передачи, хранения и отображения информации.

Основная задача SCADA --- обеспечить такой ЧМИ, который позволит получить полноту и наглядность представляемой на экране информации, доступность рычагов управления, удобство пользования подсказками и справочной системы, в конечном счете --- повысить эффективность взаимодействия диспетчера с системой и свести к минимуму его критические ошибки при управлении. Помимо этого, концепция SCADA позволяет решить такие задачи, как сокращение сроков разработки проектов по автоматизации и прямых финансовых затрат на их разработку.

Наибольшее значение при внедрении современных систем диспетчерского управления имеет решение следующие задачи:

\begin{enumerate}
    \item Выбор SCADA-системы исходя из требований и особенностей технологического процесса;
    \item Кадровое сопровождение.
\end{enumerate}

Выбор SCADA-системы представляет собой трудную задачу, усложненную невозможностью количественной оценки ряда критериев из-за недостатка информации.

\section{Компоненты систем контроля и управления}
