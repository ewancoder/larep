%| Namings: All captions in titles? (and in TOC)
%|Should TOC be in TOC and anyway, what should be in TOC
%| How to name PARTS in russian (hist,teor,pract)

\documentclass[a4paper,14pt]{extarticle} %class for 14pt fonts | Times New Roman 14pt
\usepackage{cmap} %for PDF copyable font
\usepackage[T2A]{fontenc} %cyrillic symbols
\usepackage[utf8]{inputenc} %T2A encoding
\usepackage[english,russian]{babel} %russian chapters
\addto\captionsrussian{\renewcommand{\contentsname}{\hfillОГЛАВЛЕНИЕ\hfill}}
\addto\captionsrussian{\renewcommand{\refname}{\hfillСПИСОК ИСПОЛЬЗОВАННОЙ ЛИТЕРАТУРЫ\hfill}}
\usepackage{pscyr} %quality fonts

\usepackage{indentfirst} %first paragraph with indent | First paragraph indent
\linespread{1.1} %1.5 interval | 1.5 interval - 1.3 linespread
\renewcommand{\rmdefault}{ftm} %Times New Roman
\frenchspacing %Only one space between words | frenchspacing (between words)

\usepackage[head=0pt]{geometry}
    \geometry{left=2.5cm} %| Geometry
    \geometry{right=1cm}
    \geometry{top=2cm}
    \geometry{bottom=2cm}

\usepackage{graphicx} %Pasting images
\usepackage[justification=centering]{caption} %Redefine fig/tab caption + centering captions by default | Table and Image (floats) are captioned at the center
    \DeclareCaptionLabelFormat{gostfigure}{Рисунок #2} %|Naming images/tables IN 1 CHAPTER: 1.1 or 1; IN ENDING: 4.1, 4 or just 1.2 or 666
    \DeclareCaptionLabelSeparator{gostsep}{~---~} %|Labelsep - long tire?
    \captionsetup{labelsep=gostsep}
    \captionsetup[figure]{labelformat=gostfigure}
    \captionsetup{figurewithin=section}
    \captionsetup{tablewithin=section}

\usepackage{subfig} %for \subfloat | Should them be centered, one-level both or else (multiple image aligning + captions aligning)
%\usepackage{subcaption} %for \subcaptionbox
\renewcommand{\thesubfigure}{\asbuk{subfigure}}

\usepackage{fancyhdr} %colontitle package
    \pagestyle{fancy} %set this style
    \fancyhf{} %clear values
    \fancyfoot[R]{\thepage} %top numbering | Top numbering HEIGHT vs all text HEIGHT
    \renewcommand{\headrulewidth}{0pt} %remove line

\usepackage{longtable} %|Should I use longtable or not?
\usepackage{multirow} %for multiROWs
\usepackage{colortbl} %for coloring table cells
%\usepackage{hhline} %for use \hhline instead of \cline, because "colortbl" package is not compatible with \cline
\usepackage{array} %for centering width-ed table cells
    \newcolumntype{C}[1]{>{\centering\let\newline\\\arraybackslash\hspace{0pt}}p{#1}}

\usepackage{amssymb} %for \land
\usepackage{amsmath} %for curly equation!

\usepackage{tikz}
\usepackage{tkz-euclide}
    \usetkzobj{all}
    \tikzset{
        right angle quadrant/.code={
            \pgfmathsetmacro\quadranta{{1,1,-1,-1}[#1-1]}     % Arrays for selecting quadrant
            \pgfmathsetmacro\quadrantb{{1,-1,-1,1}[#1-1]}},
        right angle quadrant=1, % Make sure it is set, even if not called explicitly
        right angle length/.code={\def\rightanglelength{#1}},   % Length of symbol
        right angle length=2ex, % Make sure it is set...
        right angle symbol/.style n args={3}{
            insert path={
                let \p0 = ($(#1)!(#3)!(#2)$) in % Intersection
                    let \p1 = ($(\p0)!\quadranta*\rightanglelength!(#3)$), % Point on base line
                    \p2 = ($(\p0)!\quadrantb*\rightanglelength!(#2)$) in % Point on perpendicular line
                    let \p3 = ($(\p1)+(\p2)-(\p0)$) in  % Corner point of symbol
                (\p1) -- (\p3) -- (\p2)
            }
        }
    }
\usepackage{circuitikz}
\usetikzlibrary{circuits.logic.IEC}
\input kvmacros %karnaugh macros

\usepackage{titlesec} %For new page with each section | Should it begin with new section
\newcommand{\sectionbreak}{\clearpage} %This is with titlesec
\titleformat{\section}{\normalfont\fontsize{16}{15}\bfseries}{\thesection}{1em}{}
\titleformat{\subsection}{\normalfont\fontsize{14}{15}\bfseries}{\thesubsection}{1em}{}
\titleformat{\subsubsection}{\normalfont\fontsize{14}{15}\bfseries}{\thesubsubsection}{1em}{}

\setlength\parindent{1.1cm}

\bibliographystyle{unsrt} %|Right way to use bibliography: urls and books
\usepackage{cite} %for [1-3], not [2,1,3] | should it be like this?
\usepackage{url} %for url citations
%\usepackage[nottoc]{tocbibind} %for bibtex in toc

\usepackage{nameref} %for named references to chapters | Should I use them?
%\renewcommand{\theequation}{\arabic{section}.\arabic{equation}}
\numberwithin{equation}{section}

%Macros commands
\makeatletter
\renewcommand{\@biblabel}[1]{#1.} %make bibliography within dot-separator |Can TOC be with such dots
\setlength{\@fptop}{0pt} %smaller offset on float top (images begin from top) |Float centering or TOP?
\setlength{\@fpsep}{8pt} %smaller offset between floats (images placed consecutively from top) |Float vfill or TOP
\g@addto@macro\@floatboxreset\centering %Manual centering all float captions
    %\g@addto@macro - LaTeX build-in macro which GLOBALLY ADDS #2 to #1
    %@floatboxreset - internal command for figure environment
%\renewcommand*\l@section{\@dottedtocline{1}{1.5em}{2.3em}} %Default line
\renewcommand*\l@section{\@dottedtocline{1}{0em}{1.3em}} %| Should TOC be with such margins?
\renewcommand{\@listI}{\topsep=0pt} %make smaller offset between list items (and minted, and etc.) |List items offset AND BETWEEN LISTS even bigger
\makeatother

\usepackage[section]{minted} %| Does listing style fits?
\renewcommand\listingscaption{Листинг}

%\usepackage{rotating} %for sidewaystable

\renewcommand{\labelitemi}{-} %For dashed itemize | Should it be DASHed?

\usepackage{enumitem} %For itemize indentation setting as GOST implied |This is PECH. said so (check both itemize and enumerate)
\setlist[itemize]{leftmargin=1em}

\usepackage{verbatim} %for \begin{comment}
\usepackage{supertabular}
